\documentclass[a4paper,notitlepage]{article}
\usepackage{ssn-format}
\title{FITS file generation flow}
\author{Atsushi Shimono}
%\date{2016--05--18}
\begin{document}

\drafttrue
\SSNID{00024}
\SSNREV{001}
\SSNCATEGORY{ICS}
\SSNChangeRecord{
}
\SSNReference{(none)}
\SSNAttachment{(none)}
\SSNWritten{Atsushi Shimono}

\ssnhead

\section{Abstract}

This note is a reference document on how PFS ICS actors collect information for 
FITS files, save as FITS files, and transfer to Gen2.


\section{Backgrounds and constraints}

PFS ICS summit system has one iSCSI storage system for file storage, with multiple
iSCSI partition by purpose, such as NFS storage or NCU up-the-ramp storage. One iSCSI
partition will be mounted by a physical host (TBD\footnote{Refer SSN-00018 for ICS 
on-site storage plan}) to be provided as NFS storage for
the entire PFS ICS summit system network. Every FITS files shall be saved into this
storage by each camera control actor both for PFS internal archive (backup) and for
the official data archive (via Gen2). Sending FITS files to Gen2 will be handled by
g2t actor or dedicated actor(s) \footnote{Gen2 does accept multiple file transfer requests
from multiple starting points}, via FTP pull or direct data transfer in a connection.
For FTP pull request, following a request post from instrument to Gen2 with a pair
of a Gen2 ID and a full path for file on a FTP server \footnote{one FTP server definition per
instrument?; TBC} for normal FITS filetransfer, or only a full path without Gen2 ID
for AG FITS image transfer, Gen2 initiates FTP connection to pull a file specified
in the request. PFS ICS will have one FTP server for this data transfer.

Gen2 has a dictionaly of status to be pushed and pulled from instruments, named
as SBR.INSROT (SBR = telescope status, INSROT = InR angle) or PFS.FRAMEID (PFS =
instrument name, FRAMEID = current frame ID like in PFSA%08d) which keeps only the
current values pushed from instruments, and any instrument can pull values pushed
from telescope or any instrument. Some FITS header keywords have --STR (start of
exposure), --END (end of exposure), --MIN (minimum within exposure; like detector
temperature), or --MAX (maximum wihtin exposure), instrument need to keep values
for them by each instrument itself. PFS has similar mechanism of a dictionaly of
status within actorcore (client library of tron), statuses are pushed from
subsystems including g2t for values from a dictionaly of status of Gen2 (assuming
periodical pull from Gen2 and push to MHS for pre-defined list of keys) and camera
actors will use them for FITS header.


\section{CCD FITS file generation by [BRN]CU}

Information required for FITS file generation are all in hand in [BR]CU, image
data is read out from CCD under control of [BR]CU, and values for FITS header are
in a dictionaly of status in actorcore. Once a FITS file successfully saved into
the NFS storage, [BR]CU actor will return its exposure call with pushing FITS
file name as a status (so any actor can get an event from actorcore library when
a target FITS file was saved).

Only interface as a sequence of detector operation will be timings, especially
with shutters.


\section{ToDo}

\subsection{Add note for NCU (H4RG), AGCC (AG camera), and MCS}

\subsection{identify a way to keep value for --STR, --END, --MIN, or --MAX}
  We may need similar routine in many actor code, might be better to have some API
for this within actorcore or shared library.

\subsection{write a formal ICD for directory organization of NFS storage}
  e.g. \verb|<mount_path>/xCUy/YYYY-MM-DD/PFSA%08d.fits|, just first \verb|<mount_path>/xCUy| part
could be enough, following part will not corride among actors.

\subsection{develop a way for target list in [BRN]CU FITS file}
  (Only?) two ways are possible: as extHDU in binary table in each image FITS files,
as one dedicated FITS file with only one binary table. IIC (or FPS) will write Cobra
positioning results as a binary table FITS file in shared NFS storage and push its
full path  as a status, [BRN]CU actor can get information required (full binary table
from FITS file for additional extHDU, or just an ID as FITS header - could be just
a %06d visit ID) and again FITS file generation process is closed within each actor. 




\end{document}

