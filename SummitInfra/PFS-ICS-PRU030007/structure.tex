\section{Structure}

\subsection{MCS, SPS and AGCC data}

There will be one ZFS file system to manage the raw data from the MCS, SPS and AGCC systems.
The subdirectories for the individual systems will have the following structure:

\begin{verbatim}
/data/raw/<DATE>/{mcs, sps, agcc}
\end{verbatim}

Where \verb!<DATE>! will have the form \verb!YYYY-MM-DD!, with \verb!YYYY! corresponding to the year,
\verb!MM! the 2-digit representation of the month, and \verb!DD! the day\footnote{Note that 
the day transition is set for UTC-10, or 2pm Hawaii-Aleutian Standard Time.} within that month. 
Example: \verb!/data/raw/2020-09-20/mcs! would be the directory holding MCS data for day 2020-09-20.

For all 3 filesystems, the user shall be \verb!pfs-data:pfs! with \verb!chmod! access 02755 . 
And in all cases the default file compression will be switched off, as FITS compression will be used.  

\subsection{DRP data}

There will be an additional file system to store DRP-related data:

\begin{verbatim}
/data/drp 
\end{verbatim}

For this file system the user shall be \verb!pfs! with \verb!chmod! access 02775 . 
Default file compression will be turned on in this case.


\subsection{Other file systems}

We may need additional file systems, for example to store log data (\verb!/data/log!), but these will be defined
in due course.
