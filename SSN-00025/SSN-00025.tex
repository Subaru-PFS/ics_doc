\documentclass[a4paper,notitlepage]{article}
\usepackage{ssn-format}
\title{No 'collision avoidance' system operation}
\author{Atsushi Shimono}
%\date{2016--05--18}
\begin{document}

\drafttrue
\SSNID{00025}
\SSNREV{001}
\SSNCATEGORY{ICS}
\SSNChangeRecord{
}
\SSNReference{(none)}
\SSNAttachment{(none)}
\SSNWritten{Atsushi Shimono}

\ssnhead

\section{Abstract}

In the course of on-going tests for Cobra characterizations, we have found that the motor maps of individual Cobras can be defined only approximately (i.e. with significant uncertainties) and they need to be updated from time to time as data are collected from processes of fiber configuration. These characteristics are considered to make the job substantially difficult for the PFI and MCS control software to minimize risks of (or ideally fully avoid) collisions without excessively complicated implementations. This proposal is, instead of allowing some collisions of Cobras in the process of fiber configuration, to simplify the software implementations by omitting intensive checks of whether collisions have happened/are happening but keeping attempts of collision avoidance in the planning stage.

In the recent hardware development and characterizations, we have found 
new foundings on operations and uncertainties such that 
the motor maps of individual Cobras have significant uncertainties and 
also are assumed to be updated from time to time as data are collected from 
operations. 
These newly found charactristics have significant impacts on strategies 
discussed and designed for Cobra operation over entire PFS software packages 
but not only MPS and FPS. 
From uncertainties of Cobra movement, detections of collisions by simulation 
in advance at target assignment by ETS are turned into impossible, which 
can only be simulated stochastically or may point unacceptable level of 
blocked region. 
Although implementation could be possible, but this makes overall strategy 
of software operation quite complex, which will make survey operation 
inefficient. 

This note is to point possible mitigations dealing with collisions of 
Cobras, with having only simple check on target assignment and detecting 
collision on the fly. 
In this note, these new foundings are summarized and described, and 
discussions which may affect to the operation of hardware from software 
point of view to make us possible dealing with new constraints and conditions. 

\section{Constraints and conditions on Cobra collisions}

\subsection{Uncertainties of Cobra motor map}

Cobra motor map is used for conversion between an angle of operation and 
real number of steps to be commanded to Cobra movement, and is registered 
as steps/degree per 3.6deg range for both forward and reverse movement per 
each Cobra. Cobra is operated by vibration movement of its piezo plates in 
high frequency, its movement (steps/degree) has large uncertainties up to 
10\% (regarding to detailed research on 'hide to dot', 10\% is maximum but 
not an average nor a sigma) both during movement and final destination. 
This makes simulation of collision avoidance quite difficult since we just 
are possible to simulate in stochastic way. 
This Cobra motor map will be updated frequently (in how much frequency need 
to be investigated both during system integration and in operation, but 
assuming one per night operation at least to check validity with logged 
history of movements during science operation), even we assume motor map 
will not change much by such semi-realtime update, this will introduce 
additional uncertainties into simulation of movement. 

\subsection{On-the-fly update of Cobra motor map}

Not only dedicated measurement data for Cobra motor map but statistics from 
normal operation will be used for on-the-fly update of Cobra motor map. This 
will require data cleansing before inputing to regeneration of Cobra motor 
map, e.g. data points on hitting hard stop, non-accurate measurement points 
such as 1st stage movement on small angle of 2nd stage (angular resolution 
could be quite bad when 2nd stage is small), nearby Cobras collided. 

\subsection{How harmful collision is}

In recent development and performance verification of Cobra, it found that 
collision does not make (unrecoverable) damage on its perzo motors, though 
it does not move on collision to another. Regarding a comment from NewScale, 
at least a few seconds of collision is fine. (Of course, it must doubltless 
not be harmful since we will use hardstop as mechanical stopper for end limit 
but not a switch for movement.) How much (or not) collision will damage the 
Cobra piezo motors or fiber arm, and how long we can keep commanding to move 
Cobra while hitting to hardstop or another Cobra, are questions which we 
need to investigate on small distributed test bench.

\subsection{Selectable 1st stage CW/CCW hardstop per each Cobra per each operation}

Recent discussions and development in CIT/JPL, direction of "origin" (or 
direction of 1st movement on Cobra configuration) of 1st stage has changed 
as selectable from CW or CCW per each operation per each Cobra. A main reason 
of this change is to make time shorter on 1st movement from each origin on 
Cobra configuration, but could have some impact on collision modes that 
nearby Cobra could have different movement direction of 1st stage. 

\subsection{Metrology camera operation and no strob mode}

Metrology camera system (MCS) is under development on its camera operation, 
but currently only continuous rolling shutter mode is supported for its 
exposure mode. Of course we will investigate further for longer exposure 
time, this continuous rolling shutter mode could be used during Cobra 
movement for configuration to detect unmoved spots. Such detection of 
unmoved spots will make us enable for on-the-fly detection of completion 
of movement per each Cobra or collision of Cobras whose spots stopped not 
at their target spots but stopped in close positions. 

Strob mode was considered to be used during Cobra position measurement, by 
taking a long exposure with moving one stage of Cobra to get all points in 
one exposure, but is not in the current base plan. Strob mode is designed 
with ~0.1s light on by high intensity, Cobra moves well more than spot size 
during one light on time, so it was designed to take arcs of Cobra movement 
in bright enough arcs with high intensity illumination of LEDs. We could use 
this strob mode in continuous exposures during Cobra movement of configuring 
but we will have arcs but not unelongated spots. So such continuous shots 
might not be used as higher resolution data for Cobra motor map nor as 
tracking of positions at each shot, but just a judgement of moving or stopped.


\section{No "collision avoidance" system operation}

This proposal is to change design of system operation as

\begin{enumerate}
  \item No check of collision during movement on target assignment.
    Check only no collision at the final positions where target objects are 
    placed.
  \item FPS decides which hardstop to be used right before commanding to origin.
    Deciding which hardstop to be used might not need to use motor map, but 
    just 180deg division is fine (TBC).
  \item FPS and MCS detemines collision for each MCS exposure.
\end{enumerate}

Actions on detection of collision among Cobra fiber tips during Cobra motrs 
active are need to be well defined, such that whether to stop Cobra movement 
for collided pairs (or even possible), or that can we issue new command (back 
to home position or move in another way for one or all in collided pair) 
while others are kept moveing. 

\section{Discussions}

\subsection{Probability map of trajectory of Cobra fiber tip}

Assuming uncertainties of Cobra motor map follow normal distribution with 
$3\sigma\sim 10\%$, and each step of movement is independent each other 
although reverberations of vibration could continue between steps, we can 
model its probability map with a generalized Wiener process
\footnote{Normally used in financial modeling especially for calculations of 
premiums for options, using geometric Brownian motion.}.
Summed probability map in a generalized Wiener process is shown as a standard 
deviation of each step times square of step numbers, so considering operation 
of Cobra motors are divided into unit steps, it will be written as a function 
of square of distance to operate. 


\subsection{Typical target pairs with high probability of collision}


\subsection{Data cleansing on Cobra motor map calculation}



\section{Action Items}

\subsection{Test using Cobra test bed with dense population}

\subsection{Collision simulation by changeable hardstop direction}


\end{document}

