\documentclass[a4paper,notitlepage]{article}
\usepackage{ssn-format}
\title{Preparation at Subaru/Summit before delivery of subsystems}
\author{PFS ICS Team}
\date{2015--11--27}
\begin{document}

\SSNID{00020}
\SSNREV{001}
\SSNCATEGORY{ICS}
\SSNChangeRecord{Rev.001 / First Release / 2015--11--2727}
\SSNReference{(none)}
\SSNAttachment{(none)}
\SSNWritten{PFS ICS Team}

\ssnhead

\section{Abstract}

This note is a summary of items to be done at Subaru/Summit before receiving 
subsystems from institutions, in electronics and software point of view. 
Aim of this summary is to show hardware need to be purchased or delivered 
at certain epochs for the project and Subaru, but will not include any 
software module (which will not have highly solid delivery date…). 
Also connections or power supplies required will be shown per each setup. 


\section{Items and assigned delivery}

Followings are hardware items which the project (or Subaru) need to prepare 
or deliver to the summit, and subsystems use these items.

PFI (PF part) and MCS will be delivered as complete hardware which does not 
need any support system delivered separated. This section lists items which 
will be placed to CB2F or SpS (TUE-IR / IR4). 

\begin{table}[htb]
\caption{List of items and required point of delivery (CB2F)}
\label{tbl:listitem-cb2f}
\begin{center}
\begin{tabular}{c||c|c|l}
Item & Prepared by & Required by & Remarks \\
\hline
Rack to host items & Subaru & before all of delivery & Assumed deepest unit 
  \~{}760mm (MPS) \\
Network switch (PFS core) & IPMU & 1st at MCS, 2nd at SCR & Already in hand \\
ICS RAID6 storge & IPMU & SM1 & MCS backup could be done with small storage? \\
VM host computers & IPMU & 1 at MCS to run MHS & Some already in hand \\
Serial communication host & IPMU & PFI & \\
KVM, PDU etc & IPMU & MCS & some already in hand \\
DB host & IPMU & SM1? & Spec. TBD 
\end{tabular}
\end{center}
\end{table}

\begin{table}[htb]
\caption{List of items and required point of delivery (SpS)}
\label{tbl:listitem-sps}
\begin{center}
\begin{tabular}{c||c|c|l}
Item & Prepared by & Required by & Remarks \\
\hline
5th rack & JHU & SCR &  \\
Network switch to CB2F & IPMU & SCR & Already in hand \\
UPS and power supply & Subaru & SCR &  \\
1U host computer at 5th rack & IPMU & SCR & Spec. TBD \\
Telemetry devices and system & TBD & SCR or SM1 & 
\end{tabular}
\end{center}
\end{table}


\section{Setups required for subsystem commissioning}

Setups at Subaru required for subsystem commissioning are listed in this 
section. Assumed points of delivery are: 
\begin{itemize}
  \item MCS at 2016/08
  \item SCR at 2017/06
  \item SM1,2 at 2017/08
  \item PFI at 2018/02
\end{itemize}

\subsection{MCS}

MCS will be installed as Cs instrument. 
We need to have minimum operation hardware at CB2F, including PFS network.

\begin{itemize}
  \item 19inch Rack (Subaru)
  \item PFS network switch (at least 1; IPMU; already in hand)
    : Also connection to subaru’s core switch
  \item VM host (IPMU; 1 already in hand)
    : Hosting administrative VM, e.g. dnsmasq
  \item KVM, PDU (IPMU; already in hand)
\end{itemize}

At this stage, no special care for power distribution is required, 
only like connecting things to PDU at CB2F. Also network is simple, 
we need two fiber connection to Cs and its standby (one active, one backup), 
one connection to subaru core switch with routing.

\subsection{SCR}

In parallel to build SCR (Spectrograph Clean Room) at TUE-IR (IR4), 
we need to have control hardware for SCR.

\begin{itemize}
  \item 2nd network switch at CB2F (IPMU; already in hand)
    : might only need 1 SFP to SpS, 
    but better to have and check working at this moment
  \item Network switch at SpS (IPMU; already in hand)
  \item 5th rack (from JHU)
  \item UPS at IR3, with power distribution/panel (Subaru)
  \item 1U host computer (also for SCR control) (IPMU)
  \item Telemetry devices (TBD)
    : e.g. Lakeshore temperature monitor and sensors
\end{itemize}

At this point, we need full power distribution system to supply 
spectrograph, both UPS and power panel w/ cables. Environment 
active control subsystem for SCR shall be considered as a part of 
SCR, so not listed here.

For network and storage, we only need “connected” at this point, 
so performance evaluation on network in parallel to SCR preparation 
(or validation) is fine.

\subsection{SM1, 2}

Before we have SM1,2 at summit, we need almost all of hardware. 

\begin{itemize}
  \item ICS RAID6 storage (IPMU)
  \item DB host (IPMU)
    : Not as full operation (e.g. interaction with ETS or survey coordination), 
    but to store MHS status archive or telemetry etc.
\end{itemize}

Before this point, PFS network and storage need to be fully prepared and 
validated on performance point of view (for NCU1,2 operation).


\subsection{PFI}

Most of computing and power resources need to be active well before delivery 
of PFI to the summit.

\begin{itemize}
  \item Serial communication (IPMU)
    : This communication will be validated at factory of melco 
      (Mitsubishi; at Amagasaki) in test setup.
\end{itemize}




\end{document}

