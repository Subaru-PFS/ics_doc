\documentclass[a4paper,notitlepage]{article}
\usepackage{ssn-format}
\title{Data rate from reading out detectors}
\author{PFS ICS Team}
%\date{2012--08--28}
\drafttrue
\begin{document}

\SSNID{00001}
\SSNREV{001}
\SSNCATEGORY{ICS}
\SSNChangeRecord{Rev.001 / First Release / 2012--08--28}
\SSNReference{(none)}
\SSNAttachment{(none)}
\SSNWritten{PFS ICS Team}

\ssnhead

\section{Abstract}

To design computer hardware system including network connections for detector 
system of PFS, we need to study how much data rate is assumed for the entire 
system. 
Estimated data rate for detectors over the entire PFS instrument is 
summarized in this study report. 

Note: In version 1, 'ideas for computer hardware system' was included in this 
study note, to be used by further studies such as design of PFS network. 
Designs of instrument control modules and their hardware have been developed 
and (partly) implemented in other documents, these parts are removed from 
version 2.

\section{Detector operational condition and estimated data rate}

\subsection{Detector operational condition}

Operational conditions are summarized as Table.~\ref{datarate-cond} (science 
detectors in SpS/xCU), and Table.~\ref{datarate-cond-other} (others
\footnote{This lists only operational cameras, but excluding support cameras 
for monitoring and/or surveillance whose data will not be archived as FITS.}).

\begin{table}[htb]
\caption{Operational conditions of science detectors}
\label{datarate-cond}
\begin{center}
\begin{tabular}{c||r|r|l}
   & CCD & IR detector & unit \\
  \hline
  Chip format & 2k x 4k (8M) & 4k x 4k (16M) & pixels \\
  Chip per dewar & 2 & 1 & chips / dewar \\
  Total dewars in system & 8 & 4 & dewars \\
  Readout channel & 4 & 32 & channels / chip \\
  Readout speed & 75kHz & 100kHz & (1000 pixels / sec) \\
  Readout per frame & 28 & 5.3 & sec \\
  \hline
  Datasize per chip & 16MB & 32MB & per chip \\
  Total datasize per frame & 256MB & 128MB & (all spectrograph) \\
\end{tabular}
\end{center}
\end{table}

\begin{table}[htb]
\caption{Operational conditions of non-science detectors}
\label{datarate-cond-other}
\begin{center}
\begin{tabular}{c||r|r|r|r|l}
Name & Chip format & \# & MB/frame & Readout & Remarks \\
\hline
AGCC (AG Camera) & 1k sq. & 6 & 2MB & FT & typ. operation in 1 to 0.1 Hz \\
FCC (Field Center Camera) & 250 sq. & 1 & 0.13MB & 44Hz & \\
FVC (Fiber Viewing Camera) & 1k sq. & 1 & 2MB & (TBC) & \\
\hline
MCS (Metrology camera) & 50M pixel & 1 & 100MB & 1.1Hz & CMOS rolling shutter \\
\end{tabular}
\end{center}
\end{table}

\subsection{Estimated data rate and operation conditions}

\subsubsection{CCD (SpS/[BR]CU)}

For CCD, since all readout will be destructive, we will readout only once 
per exposure. This readout will occur just after its exposure.

Total data size is 256MB per exposure which will be readout in 28.0 sec, 
and estimated data rate is 73 Mbps in total (all spectrographs) from SpS 
switch to PFS storage. 
Data transfer from PFS storage to Gen2 will occur right after FITS files are 
finalized by SpS/[BR]CU, its total data volume is 256MB (2Gbits). 


\subsubsection{IR detector (SpS/NCU)}

For IR detector, we will use two methods: Fowler sampling, up-the ramp. 
Still by data rate point of view, these two are the same -- we can concentrate 
on the maximum data rate. This ignores costs for real time data processing for 
finalizing as one data frame, though. 

The maximum data rate is one for up-the ramp sampling, which is continuous 
data flow with the specified readout speed. 
With the readout speed of 100 kHz and using 32 channels, one NCU generates 
51.2Mbps (6.4MB/s), and 5.76GB for 900sec exposure. 

Data transfer from SpS to PFS storage (at CB2F) will be \~{}200Mbps for one 
stream from all spectrographs \footnote{PFS may have up to two stream for 
each up-the-ramp data, one is for one FITS file per one full frame readout, 
and another is for stacked FITS file over one exposure. This can make total 
data transfer to be double over entire data readout during exposure.}. 

Data transfer from PFS storage to Gen2 has two parts: 
\begin{itemize}
  \item On the fly FITS file transfer with single up-the-ramp frame during 
    exposure, for health check of instrument during exposure
  \item FITS file for entire exposure with multiple (or all of) up-the-ramp 
    frame transferred after exposure finished, for archive of science data
\end{itemize}

For 1st part, a set of FITS files (4 files) of single up-the-ramp readout 
from all 4 NCUs will be transferred, and their total data size is 128MB 
(or 256MB if PFS handles real pixels and references in one pack). Minimum 
separation between two transfer is 5.3sec which is a time for full frame 
readout, but interval of sending to Gen2 is not defined for real operation. 
128MB can be transferred about 1sec with 1Gbps connection. 

For 2nd part, data size to be transferred depends on exposure time (total 
number of frames from up-the-ramp) and sampling rate (all up-the-ramp frame 
or picked up). For typical exposure time of 900 sec and all up-the-ramp frame, 
total data size (of 4 FITS files) is 23GB (184Gbits) for all spectrograph 
modules. 

\subsubsection{AGCC --- AG camera}

AG camera control computer is at CB2F with dedicated connection of PCIe bus 
extender to six AG cameras at PFI Cobra bench. No load to ethernet connection 
between CB2F and PFI for AG camera data transfer. 
Maximum data rate from six AGCC to PFS storage through AG camera control 
computer at CB2F, which flows only on PFS network switch at CB2F, 
is \~{}100Mbps (2MB x6 in 1Hz), 
for assumed normal operation of 2 cameras in 1Hz (for AG) and 1 camera in 
0.1Hz (for focusing) data rate is \~{}34Mbps. 

During AG running, two image transfer will run: 
\begin{itemize}
  \item Image to V-LAN, 512 pixels sq. with 16bit (0.5MB each) with maximum 1Hz.
    For caching (data transfer frequency to V-LAN could be different from one 
    of AG camera readout) and debug (to be checked later) purpose, data sent 
    to V-LAN could be saved into PFS storage. 
  \item FITS file to Gen2 for operational display purpose but not to be 
    archived. This will be just transfer FITS files saved by AGCC. 
\end{itemize}

PFS is planning to archive guide star images of auto guiding, but details are 
not fixed yet. Maximum data size for assumed normal operation (2 cameras in 
1Hz for AG and 1 camera in 0.1Hz for focusing) and typical exposure time of 
900sec is \~{}3.8GB. 


\subsubsection{FCC --- Field center camera}

This camera is assumed to be used only on commissioning or special case of 
instrument health check.
This camera will not generate traffic during normal operation. 


\subsubsection{FVC --- Fiber viewing camera}

This camera is mounted at backside of fixed fiducial fibers in parallel to 
fixed fiducial back illuminator lamp system, and looks sky through fixed 
fiducial fibers, which is assumed to be used only on commissioning or special 
case of instrument health check. 
This camera will not generate traffic during normal operation. 

\subsubsection{MCS --- Metrology camera}

Metrology camera detector is operated in a rolling shutter mode of CMOS image 
sensor. As a system, MCS (Metrology camera system) has a mechanical shutter 
to control exposure time, and will stack into one output (single image is 
100MB, not yet defined for stacked image) for one exposure. 
Control computer of this camera is at Cs container, and images will be saved 
into local storage of the control computer. MCS is used during Cobra 
configuration between exposures (for normal operation), with around 5 to 10 sec 
interval and upto 10 times per configuration. 

The last exposure is assumed to be transferred to Gen2 through PFS storage, 
but not yet defined for other exposures. 


\section{Estmated data rate for operation modes}

This section is to summarize estimated data rate among PFS subsystems, PFS 
CB2F NFS server, and Gen2 by operation modes, such as on exposure, on CCD 
readout. 
PFS iSCSI storage server and NFS server will be installed into the same rack 
at CB2F, and will be connected through the same network switch (if we use 
ethernet iSCSI but not FC connection option
\footnote{Selection is just to replace SFP inserted into the PFS storage 
server.}
), their configurations are relatively easy to be changed than other 
connections like CB2F to PFS subsystems. 
In this section, data flow to the PFS NFS server is taken but not to 
PFS iSCSI storage server (through PFS NFS server). 

\subsection{Possible data flow by each operation mode}

Between two exposure, PFS has two modes for its operation -- execute next 
exposure with or without Cobra configuration
\footnote{There could be one another mode to slew telescope to new field 
before Cobra configuration, but not much change on data flow except for MCS 
data.}. Former is normal operation, and latter is used especially for some 
survey program with 2x 450sec exposure for total 900sec integration per one 
target assignment. 
When next exposure is commanded without Cobra configuration, MCS will not be 
used and no data flow from MCS happens, and AG camera operation continues 
over between two exposures. 
Also data stream to Gen2 from PFS storage continues after next exposure 
starts, CCD data transfer will be in parallel for (most of) all, and 
IR up-the-ramp data (11.5GB for 450sec exposure) will not finished in 28sec 
(11.5GB in 28sec is \~{}3.3Gbps). 

Once exposure ends with shutter closed, following operations are executed 
in parallel:
\begin{itemize}
  \item FITS files for IR detectors are created and starts sending to Gen2
    (continuous)
  \item CCD readout (for first 28sec), and sending to Gen2 follows
  \item Cobra configuration (configuration takes \~{}100sec, MCS image for 
    final assignment taken after configuration)
\end{itemize}

If there is no image data flow from Cs (MCS) for first 28sec after shutter 
closed
\footnote{As for now, this assumption is true that PFS assumes to send only 
the final image to PFS storage during operation.}, 
data flow configuration is the same for all operation types except for AG. 

\subsection{Summary}

Estimated data rate for operation modes are summarized as 
Table.~\ref{datarate-modes}, taking typical exposure time of 900sec. 
Data rate is shown in two types, one is in total size for transfer between 
two storages which can vary by network conditions and configurations, 
one is in Mbps for fixed data outputs. 
At the top of table, data flow origin and destination is listed, 
'CB2F' means PFS NFS server for storage operation, 
and 'PFI to CB2F' means PFI control computer at CB2F to 'CB2F'.
Data rate for each flow is shown in a column of connection (at the top of 
table). 


\begin{table}[htb]
\caption{Estimated data rate for operation modes}
\label{datarate-modes}
\begin{center}
\begin{tabular}{c||r|r|r||r|r||l}
     & \multicolumn{3}{c||}{To CB2F} & \multicolumn{2}{c||}{From CB2F} &  \\
Name & SpS & Cs & PFI & Gen2 & V-LAN & Remarks \\
\hline
\hline
\multicolumn{7}{c}{During exposure} \\
\hline
IR up-the-ramp & 200Mbps & -- & -- & $<$ 200Mbps & -- & To Gen2 for monitoring \\
AG & -- & -- & 34Mbps & $<$ 30Mbps & -- & To Gen2 for monitoring \\
V-LAN & -- & -- & -- & -- & 0.5Mbps & \\
(Science data) & -- & -- & -- & ($<$ 13GB) & -- & (for without Cobra configuration) \\
\hline
Total & 200Mbps & -- & 34Mbps & $<$ 230Mbps & 0.5Mbps &  \\
\hline
\hline
\multicolumn{7}{c}{First 28sec after exposure (shutter closed) with Cobra 
  configuration} \\
\hline
IR data & -- & -- & -- & (23GB) & -- & Size depends on exposure time \\
AG data & -- & -- & -- & (3.8GB) & -- & Size depends on exposure time \\
CCD readout & 73Mbps & -- & -- & -- & -- & \\
\hline
Total & 73Mbps & -- & -- & (26.8GB) & -- &  \\
\hline
\hline
\multicolumn{7}{c}{First 28sec after exposure (shutter closed) without Cobra 
  configuration} \\
\hline
IR data & -- & -- & -- & (23GB) & -- & Size depends on exposure time \\
AG data & -- & -- & -- & (3.8GB) & -- & Size depends on exposure time \\
CCD readout & 73Mbps & -- & -- & -- & -- & \\
AG & -- & -- & 34Mbps & $<$ 30Mbps & -- & To Gen2 for monitoring \\
V-LAN & -- & -- & -- & -- & 0.5Mbps & \\
\hline
Total & 73Mbps & -- & 34Mbps & (see above) & 0.5Mbps &  \\
\hline
\hline
\multicolumn{7}{c}{After CCD readout finished before shutter open} \\
\hline
IR data & -- & -- & -- & (23GB) & -- & Size depends on exposure time \\
AG data & -- & -- & -- & (3.8GB) & -- & Size depends on exposure time \\
MCS data & -- & (100MB) & -- & (100MB) & -- & \\
\hline
Total & -- & (100MB) & -- & (26.9GB) & -- &  \\
\hline
\hline
\end{tabular}
\end{center}
\end{table}




\end{document}

