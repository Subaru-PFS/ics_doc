\documentclass[a4paper,notitlepage]{article}
\usepackage{ssn-format}
\title{Data rate from reading out detectors}
\author{PFS ICS Team}
\date{2012--08--28}
\begin{document}

\SSNID{00001}
\SSNREV{001}
\SSNCATEGORY{DCR, IIC}
\SSNChangeRecord{Rev.001 / First Release / 2012--08--28}
\SSNReference{(none)}
\SSNAttachment{(none)}
\SSNWritten{PFS ICS Team}

\ssnhead

\section{Abstract}

To design computer hardware system including network connections for detector 
system of PFS, we need to study how much data rate is assumed for the entire 
system. 
In this study report, we will summarize estimated data rates for detector 
systems and perform some trade-off study of existing (consumer) 
hardware system. 


\section{Operational condition and estimated data rate}

\subsection{Operational condition}

Based on the CoDR Technical Document of the PFS project, 
operational conditions are summarized as Table.~\ref{datarate-cond}.

\begin{table}[htb]
\caption{Operational conditions of detectors}
\label{datarate-cond}
\begin{center}
\begin{tabular}{c||r|r|l}
   & CCD & IR detector & unit \\
  \hline
  Chip format & 2k x 4k (8M) & 4k x 4k (16M) & pixels \\
  Chip per dewar & 2 & 1 & chips / dewar \\
  Total dewars in system & 8 & 4 & dewars \\
  Readout channel & 4 & 32 & channels / chip \\
  Readout speed & 75kHz & 300kHz & (1000 pixels / sec) \\
  \hline
  Datasize per chip & 16MB & 32MB & per chip \\
  Total datasize per frame & 256MB & 128MB & (all spectrograph) \\
\end{tabular}
\end{center}
\end{table}


\subsection{Estimated data rate for CCD}

For CCD, since all readout will be destructive, we will readout only once 
per exposure. This readout will occur just after its exposure.

Total data size is 256MB per exposure which will be readout in 28.0 sec, 
and estimated data rate is 73 Mbps in total (all spectrographs).


\subsection{Estimated data rate for IR detector}

For IR detector, we will use two methods: Fowler sampling, up-the ramp. 
Still by data rate point of view, these two are the same -- we can concentrate 
on the maximum data rate. This ignores costs for real time data processing for 
finalizing as one data frame, though. 

The maximum data rate is one for up-the ramp sampling, which is continuous 
data flow with the specified readout speed. 
With the readout speed of 300 kHz, 9.6M pixels per sec per detctor will be 
read, and this corresponds to data flow of 153.6 Mbps per detector. 
In total (4 detectors in 4 spectrograph), required data bandwidth is 
614 Mbps.

\subsection{Network connection for component control}

From data rate point of view, required bandwidth of network connection for 
detector system (and spectrograph system) is quite low amount comparing to 
the other connections described in previous sections. 


\section{Candidates of computer hardware system}

\subsection{Network connection}

Currently, since we have not yet fixed our control hardware topology design, 
we cannot fix network connection requirement. 
Since the most strict requirement on network connections will be lines 
between the IR 3rd floor of the telescope dome and the conputer room at the 
control building, this section will concentrate on this point. 

Assuming direct 1Gbps fiber connection between readout electroncis system 
and data storage computer, effective data rate will be 500Mbps or so per 
connection. 
For final images without up-ramp sampling of IR detectors, 
data rate in total will be less than this. 
To include up-ramp sampling of 614Mbps, 
we might be better to have multiple (at least two) 1Gbps fiber connection 
considering safety buffer and worst case, such as having one 1Gbps fiber 
connection per two spectrographs and two in total. 
Since up-ramp data flow is only during exposure, we can share these connections 
for CCD readout. 


\subsection{Temporary storage}

For storage side, (continuous writting of) 0.6Gbps is not a problem. 
We can have many options, including

\begin{itemize}
  \item Local temporary storage (SSD etc.) per each data storage computer
  \item Large and high performance data storage server shared by entire system
\end{itemize}

To make the final hardware design (or its proposal), 
we need to perform trade-off studies more in detail, including further back-end 
systems such as a FITS handling system. 


\end{document}

