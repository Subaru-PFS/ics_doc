\section{Test Case SPS-ALERT-CTHIGH-010: Cooler Temperature Too High}

\subsection{Description}

Checks that expected alarm is raised when the cooler temperature exceeds the allowed upper limit.

\subsection{Pass/Fail Condition(s)}

\begin{description}
\item [Pass] Expected alert seen on GUI and written to log when temperature is above the allowed limit.
\item [Fail] Any result otherwise. For example:

\begin{enumerate}
    \item The expected alert is {\it not} raised when the temperature exceeds the limit.
    \item An unexpected alert raised when temperature is in fact {\it within} the allowed range.
\end{enumerate}
\end{description}

\subsection{Hardware constraints}

Cryostat is cooled?

\subsection{Initial conditions}

SM module and temperature board deactivated. AlertsActor deactivated.

\subsection{Procedure}
\label{sec:alerts-proc}

\begin{enumerate}
    \item Set temporary range override: \texttt{alerts override key=coolerTemps field=0 limits=null,20} 
    (see notes below)
    \item Start alerts actor and SM module with temperature board activated
    \item Monitor alerts
    \item Restore nominal temperature range: \texttt{alerts dropOverride key=coolerTemps field=0}
\end{enumerate}

\subsection{Additional Notes}
The temperature range is temporarily overridden to allow the alarm to be triggered without 
having to force a hardware change - ie., a change to the ambient temperature.
