\documentclass[a4paper,notitlepage]{article}
\usepackage{ssn-format}
\title{List of ICS subsystems and its controllable hardwares}
\author{PFS ICS}
%\date{201X--XX--XX}
\begin{document}

\drafttrue
\SSNID{00000}
\SSNREV{000}
\SSNCATEGORY{ICS}
\SSNChangeRecord{}
\SSNReference{(none)}
\SSNAttachment{(none)}
\SSNWritten{}

\ssnhead

\section{Abstract}

This is not well organized ICD, but is written to have mutual understanding 
for entire PFS ICS subsystems kept updated with hardware development. 
Each interface shall be defined as a separated ICD in detail, and link to 
them will be included in this document. 

For this purpose, this document will contain following items: 
\begin{itemize}
  \item brief description per each subsystem
  \item summary of controllable hardware instances
  \item relationship to other subsystems, including database and storage
\end{itemize}

\section{Summary: List of subsystems and relations}

\subsection{Summary list}

Table.~\ref{summary-list-subsystem} 
shows list of subsystems, 
place and category (detector, motor etc) of its hardware, 
and its relations to common services (database, storage). 

\begin{table}[htb]
\caption{Summary of responsibilities on coordinate transformation}
\label{summary-list-subsystem}
\begin{center}
\begin{tabular}{c||c|c||c|c||l}
Name & Place & Category & DB & Storage & Note \\
\hline
IIC & Ctrl & --- & x & x &  \\
MHS & Ctrl & --- &   &   &  \\
Environment (Status) archive database service & Ctrl & --- &   &   &  \\
Event logger database service & Ctrl & --- &   &   &  \\
Detector control and readout system &  &  &   &   &  \\
Vacuum and Cryogenic Control System &  &  &   &   &  \\
Spectrograph temperature monitoring system &  &  &   &   &  \\
Mechanism control system &  &  &   &   &  \\
PFICS &  &  &   &   &  \\
Acquisition and Guidance camera system &  &  &   &   &  \\
(Fixed) Fiducial fiber back illuminating system &  &  &   &   &  \\
MCS &  &  &   &   &  \\
\end{tabular}
\end{center}
\end{table}

List of places (of target hardware):
\begin{description}
  \item[Ctrl] Control building only, means no dedicated hardware
  \item[PFI] Prime Focus Instrument
  \item[SpS] Spectrograph floor
  \item[Cs] Cassegrain focus (Metrology camera container)
\end{description}

\section{Subsystem definitions without hardware}

This section lists control subsystem which does not have its hardware, 
like database system or logging system. 

Normally, from the Subaru telescope (or SOSS) side, ICS or IIC will be visible 
as so called 'OBCP'. Also, except for some required direct connections to MLP, 
this ICS or IIC will the only peer from the PFS instrument control software. 
These connections are summarized in Figure.~\tbd.


\subsection{ICS --- Instrument Control System}

This is not a subsystem but 
ICS includes everything of the PFS instrument control software systems 
including every subsystem 
control software. ICS will include any required engineering software 
including graphical user interfaces, command-line user interfaces, or 
network data API, but will not include any connected systems at the SSF 
such as an on-site data reduction pipeline or a targeting software 
instance. 


\subsection{IIC --- Instrument Interface and Controller}

IIC has two roles: 
the only interface of the ICS to the Subaru telescope control system 
including Gen2, TCS, MLP1 and any other possible candidates; the master 
controller and sequencer for the entire ICS.

\subsection{MHS --- Messaging Hub Server}

MHS is the only central communication hub 
system in entire ICS and every controllable subsystems and environment 
monitoring systems will connect to this communication hub for sending or 
recieving commands to peers, sharing status messages, arising operational 
or environmental alerts, etc.

MHS will be the only central messaging exchange system for entire PFS 
instrument including every controllable subsystems. 
To coordinate instrument control in a proper way, 
following requirements are applied to a selection of PFS MHS. 

\begin{itemize}
  \item MHS should be a star-liked message transfer and delivery system, 
    and does not require any (messy) multi-point to multi-point connection.
  \item MHS should have a capability per each peer to set / select 
    a list of acceptable peers, from which commands should be accepted and 
    otherwise should be rejected, as a configuration. 
    This could be implemented at the MHS side or at peers' side as a wrapper 
    layor; both could be accpetable.
  \item MHS should have a capability to attach external log servers as its 
    peers for every messages through the MHS. 
  \item MHS shuold have a capability to set a 'blocking flag' by each peer 
    to specify peers whose commands should be acceptable, on demand.
\end{itemize}

For a 'blocking flag', we need to be careful on the system deadlock, 
especially while defining interfaces. 


\subsection{Environment (Status) archive database service}

This service does not have controllable hardware, and will act as a sub 
service module of IIC. 
This service will receive status update messages from each subsystem, 
archive these statuses with each timestamp into database server, 
and forward status to the SOSS status server in the defined protocol. 
%
\footnote{In the previous document this was noted as following, but need to 
be reconsidered: 
This will not work on submitting requests of statuses to SOSS from PFS ICS 
including subsystems (via MHS).}

\subsection{Event logger database service}

This service is an archiver of every message exchanged through MHS. 

This service itself does not have any relationship with other systems 
(DRP or targeting software), 
but its database server should be accessible from these other systems.
\footnote{Really??}


\section{Subsystem definitions with hardware at Spectrograph floor}

Note: this part is based on LAM document (need to update)

\subsection{Detector control and readout system}

Including detector readout, detector temperature sensor and heater.

\paragraph{Definition and subsystem hardware}
DCR will include every controllable hardware within dewar and detector systems, 
which will include detector movement (tilt?) mechanisms, cryostat vacuum and 
cooling system. 
\paragraph{Peers required to communicate}
DCR should receive commands from IIC, and should transfer retrieved data from 
detector subsystems to the data archive system (SOSS Data Server; STARS) of 
the Subaru telescope through IIC. 
\paragraph{Remarks}
Considering amount of controllable hardwares, this subsystem could be divided 
into two or more subsystems: detector control and any others. 


\subsection{Vacuum and Cryogenic Control System}

Including ion pump, pressure gauge, cryo cooler, temperature sensor.

\subsection{Spectrograph temperature monitoring system}


\subsection{Spectrograph Mechanism control system}

including dithering stage, shutter and back illumination, red grating 
exchange mechanism.

\subsubsection{SMC --- Shutter mechanisms controller}
\paragraph{Definition and subsystem hardware}
SMC will control shutter mechanisms (one per each spectrograph) to open or 
to close. 
\paragraph{Peers required to communicate}
SMC should receive commands from IIC. 
\paragraph{Remarks}
This shutter is a mechanism to: 
\begin{itemize}
  \item control an exposure time by shielding detectors from fiber outputs
  \item shield detectors from light source for fiber back illumination
  \item back illuminate science fibers by reflecting light with a diffuse 
    reflector mounted on shutter plate
\end{itemize}
Since this shutter is the only mechanism to shield detectors from fiber and 
light source for fiber back illuminator, SMC and its parent control system 
(assuming IIC) should control: 
\begin{itemize}
  \item OPEN during exposure
  \item CLOSE during detectors' reading out and light source illumination
\end{itemize}

\subsubsection{FIS --- Fiber back illuminating system}
\paragraph{Definition and subsystem hardware}
FIS will turn on/off and adjust brightness of back illuminating light source 
for science fibers. 
\paragraph{Peers required to communicate}
FIS should receive commands both from IIC and PFICS. 
\paragraph{Remarks}
Since back illuminating light source for science fibers should not be turned 
on while shutter is open (except for engineering or test purpose), IIC should 
coordinate to block control of FIS depending on a status of SMC. 


\section{Subsystem definitions with hardware at PFI}

Note: PFI part by JPL+CIT is collected into PFICS

\subsection{PFICS --- PFI control software}

Including MPS (COBRA positioner movement planning software) as its 
subsystem.

\subsubsection{PFICS --- Prime Focus Instrument Control Software}
\paragraph{Definition and subsystem hardware}
PFICS will also be a master controller for everything related with COBRA 
positioners, and this should work in close collaboration with MPS. 
Also, since this includes 'movement planning and sequencing', PFICS is 
required to issue commands to required subsystems for alignment of 
positioners, including turning on and off FIS, getting list of measured 
positions from MCS. 
\paragraph{Peers required to communicate}
PFICS should receive commands from IIC, and should send commands to all related 
external subsystems like MCS and FIS, and to all subsystems under this PFICS 
like MPS. 
Also, these communication interface design should consider of on-site 
engineering operations from IIC (or the Subaru). 
\paragraph{Remarks}
This is a CIT part of (so called) COBRA positioner controller, and also 
includes communication interface between IIC and controllable PFI parts (and 
their environment monitoring). 

\subsubsection{MPS --- COBRA Positioner movement planning software}
\paragraph{Definition and subsystem hardware}
Everything related to controlling positioners themselves.
\paragraph{Peers required to communicate}
MPS should receive commands from PFICS. 
\paragraph{Remarks}
This is a JPL part of (so called) COBRA positioner controller.


\subsection{Acquisition and Guidance camera system}

\subsubsection{AGC --- Acquisition and Guidance camera system}
\paragraph{Definition and subsystem hardware}
AGC should control Acquisition and Guidance camera system to meet 
requirements from procedures of both Acquisition and Auto 
Guiding. 
Also, for engineering, integration and testing purpose, AGC might need to work 
in cooporate with PFICS. 
\paragraph{Peers required to communicate}
AGC should receive commands from IIC, and should transfer measured guide 
signal (target field positioning error for Acquisition and Guidance per 
request, guider error for Auto Guiding in commanded rate) back to IIC. 
IIC should apply coordinate transformation and transfer to suitable peer, 
such as OCS or MLP1. 

(TBD) Also, if requested by an operator, 
AGC should transfer retrieved raw data image to a handling system (VGW). 
\paragraph{Remarks}
Especially for Auto Guiding feature, data transfer should work in strict 
timing. 
AGC might be required to connect and to work in cooporate with IIC in deeply. 


\subsection{(Fixed) Fiducial fiber back illuminating system}

\subsubsection{FFIS --- Fixed fiducial fibers' back illuminating system}
\paragraph{Definition and subsystem hardware}
FFIS will turn on/off back illuminating light source for fixed fiducial fiber. 
This illuminating light source is assumed to be mounted in PFI. 
\paragraph{Peers required to communicate}
FFIS should receive commands from PFICS. 
FFIS might receive blocking request from IIC, especially during exposure 
(to reduce failure mode). 
\paragraph{Remarks}
Back illuminating light source for fixed fiducial fibers should not be turned 
on while exposure is ongoing. 
PFICS (or IIC) should coordinate to block control of FFIS. 


\section{Subsystem definitions with hardware at Cs}

\subsection{MCS --- Metrology camera system}

MCS should control its detector system, camera lens system, and (narrow band) 
filter system. Also from data retrieved from its detector system, MCS should 
pick spots and measure their centroids. 

\subsubsection{Peers required to communicate}
MCS should receive commands both from IIC and PFICS. 

(TBD) Also, if requested by an operator, 
MCS should transfer retrieved raw data image to the data archive system 
(STARS) or a handling system (VGW). 

\subsubsection{Remarks}
Although, the data format used for external interfaces will not be raw 
images but be list of (x, y) values for spots, having raw images may be useful 
to check on error (failure mode both detected and non detected) or later 
with using another pipeline etc. 


\section{Possible additionals on subsystem definitions}

Following subsystems are proposed, but yet fixed as final plan. 

\subsection{Slit viewing camera system}


\subsection{Fiber monitoring system}


\section{Matrix between control subsystems and controllable hardwares in product tree}

Table.~\ref{matrix-control-pt}
represents the matrix between software control subsystems and 
hardware product tree.

\begin{table}[htb]
\caption{Matrix between software and hardware}
\label{matrix-control-pt}
\begin{center}
\begin{tabular}{l|l||c||l}
Tree & target hardware & Software & Note \\
\hline
\end{tabular}
\end{center}
\end{table}







\section{Terminology}

\begin{description}
  \item[ICS] Instrument Control Software
\end{description}


\end{document}
