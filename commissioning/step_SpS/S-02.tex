%%%%%%%%%%%%%%%%%%%%%%%%%%%%%%%%%%%%%%%%%%%%
\subsubsection{Characterization of SpS [Daytime]}\label{secflow:SpSchar}

Using arc lamps and continuum lamps, we characterize the SMs.
Here we will use Dummy Cable B module [see DummyCableB-MainDocument.pptx (PFS-SpS:01680) for details] which is used for the integration by LAM\footnote{To achieve characterization well, we need lamps with large wavelength coverage. We also need black body lamp, but should input the light from slit or GANG connector.}.

\paragraph{PSF and spectral distribution measurement}
With arc-lamp image and continuum lamp image, we measure PSF and spectral distribution of various fibers on the detectors, respectively.
These information of the fiber image is feedbacked to DRP development.
Configuration of measured fibers is TBD\footnote{LAM will use 4 fibers with the option of extra 7 fibers sparsely arranged in the detectors, but is it acceptable for DRP development?
It seems some room for fiber configurations.
1--2 fibers per block, for instance,  are too much?
At least, the PSFs and spectral distribution both centre and edge seems required.
Besides, how many spots is needed in wavelength direction?
}.

\paragraph{wavelength coverage, spectral resolution, and throughput}
Using the arc-lamp image, we estimate wavelength coverage and spectral resolution (including wavelength dependence), and compare with specifications, by measuring the FWHMs of the spots.
Note that the wavelength coverage and the spectral resolution will be partly measured by LAM\footnote{Validation Roadmap; $\lambda$ coverage --- 2.5.3: checked by centring the image. $R$ --- 2.5.4: extrapolated by FWHM of the spot for testing image quality (maybe only center?).}.

The requirement for the wavelength coverage ({\tt REQ-SpS-41 (Analysis)}) is 380--650 nm (blue), 630--970 nm (red), and  940--1260 nm (nir).

The requirement for the spectral resolution, on the other hand, ({\tt REQ-SpS-42 (Analysis)}) is $>$2300 @ 520 nm (blue), $>$ 2800 @ 810 nm (red, LR),  $>$ 5000 @ 810 nm (red, MR), and $>$ 4100 @ 1110 um (nir).

\smallskip

Using a black body lamp, we will measure throughput of the SpS itself and compare with specifications, which is estimated by combining the performance of each element (LAM).
If measured throughput is inconsistent with analysed one, consider possible causes (vignetting, background etc.) 
The stability of the black body lamp (TBD) is needed.

The requirement for the throughput of SpS ({\tt REQ-SpS-48 (Analysis)}) is in Table \ref{tab:SpSthroughput}:
\begin{table}[!ht]
\label{tab:SpSthroughput}
\begin{center}
\caption{}
\begin{tabular}{c|cc}  \hline
 & \multicolumn{2}{c}{Throughput [ \% ]} \\
wavelength &  requirement & goal \\ \hline \hline
380 & 14 & 19 \\
440 & 38 & 49 \\
550 & 39 & 50 \\
650 & 33 & 50 \\
790 & 42 & 56 \\
980 & 36 & 47 \\
1260 & 35 & 45 \\  \hline
\end{tabular}
\end{center}
\end{table}

\begin{itembox}[l]{\suctitle{Success Criteria}}
SpS is characterized. These characters is consistent with expected by specification.
\end{itembox}