%%%%%%%%%%%%%%%%%%%%%%%%%%%%%%%%%%%%%%%%%%%%
\subsubsection{Performance Verification II [Night-time]}\label{secflow:PV2}

In the previous commissioning process (\ref{secflow:PV1}), basic instrument parameters and performance will be revealed. After this phase, a normal observational sequence can be done properly using almost the same software interface used in actual survey observations including data reduction pipeline. 

In this this commissioning phase, we will stabilize the instrument operation and the performance optimizing the softwares and the operation processes for long-term survey observations. Specifically, we will conduct the following verifications:

\begin{itemize}
\item Long exposure of faint galaxies and stars
\item DRP optimization using a wide variety of galaxies and stars 
\item Confirmation that there is no time change of the performance
\end{itemize}



\begin{itembox}[l]{\suctitle{Success Criteria}}
\begin{itemize}
\item Confirm that observation process can be done properly and stably
\item Confirm that there is no time change of the instrument performance
\item Confirm that a sufficient number of galaxies and stars was observed to optimize the 1D-DRP using a machine learning technique
\end{itemize}
\end{itembox}


