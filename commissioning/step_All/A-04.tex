%%%%%%%%%%%%%%%%%%%%%%%%%%%%%%%%%%%%%%%%%%%%
\subsubsection{Second-Pass Distortion Map [Night-time]}\label{secflow:raster}

After the process \ref{secflow:1stDM}, the science fibers should be pointed to the targets with the accuracy of $\sim 50$ um (at least less than 100 um).
In this step, we will do raster scan to improve the distortion map and achieve the fiber pointing accuracy of $\sim 10$ um RMS.

The process is as follows:
\begin{enumerate}
\item Slew the telescope to the field where enough bright stars is cataloged with good astrometry (see section \ref{sec:RasterField}).
\item Put these bright stars on the science fibers and take fiber spectra with the telescope moving in 3$''$ $\times$ 3$''$ grid, approximately three time as large as the fiber core diameters.
Note that the raster with 5 position of hexagonal pattern provide denser scan than the grid (TBD).
\item Calculate the brightest position by comparing the intensity of these spectra.
\item Repeat 1.--3. for several fields at various telescope positions, and calibrate the distortion map.
\item Check the dependency of the map on wavelength, and check if ADC works correctly, by comparing the observed positions with predicted ones by ADC.
\end{enumerate}
Calibrations will include the following factors:
\begin{itemize}
\item Telescope elevation, and azimuth
\item Temperature
\item Focus
\item ROA
\item Error in the relative position of A\&G cameras to fiber positioner.
\item Error in Cobras' position measured prior to integration at Subaru
\item Error in Catalogue
\item ADC
\item etc…… (TBD)
\end{itemize}

Assuming we take 120- --150-second exposure for a given position, it will take $\sim$ 30 minutes including slewing, fiber configuration, and acquiring sky frames.
We will take 2 sets of 4 El $\times$ 4 RoA positions ($\sim$ 8 hours) to study the dependence of the above parameters.
We repeat 2 times of these sets using SMs 1--3 (runs. 4 and 5).
Note that fibers of each SM are arranged on the focal plane to cover entire field of view, we can study distortion without SM4.
However, we will check and finalize the distortion map using all SMs at run 6.
In total, 5 nights are required in this process.

When the second distortion map is made, update ETS using this map, and/or update the repository for coordinate transformation (TBC).

\paragraph{Designated Tool for This Process}
A tool for measuring the brightest position from raster scan data.

\begin{itembox}[l]{\suctitle{Success Criteria}}
Second distortion map is obtained with error less than 10um.

\bluetext{Required long time to analyze the data?: Yes. \\
--- It takes time ($\sim$ one month) to improve distortion map, by analyzing raster scan data in various condition.
Also it needed more than one run to check dependency of temperature and so on.
}
\end{itembox}

When this step is completed, the fibers shall point to the target within the accuracy of 10 um.
Then, we verify the instrument performance in the following steps. 