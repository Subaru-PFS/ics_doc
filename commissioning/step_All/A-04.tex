%%%%%%%%%%%%%%%%%%%%%%%%%%%%%%%%%%%%%%%%%%%%
\subsubsection{Second-Pass Distortion Map [Night-time]}\label{secflow:raster}

At this step, the science fibers should be pointed to the targets with the accuracy of $\sim 50$ um (at least less than 100 um).

The goal of the accuracy of pointing of the fibers is $\sim 10$ um, which is achieved by raster scan observation in this step.

We slew the telescope to the field where enough stars is catalogued with good astrometry (see section \ref{sec:RasterField}).
We put these bright stars on the science fibers and do telescope raster scan observation by 3$''$ $\times$ 3$''$ grid, approximately three time as large as the fiber core diameters.
\redtext{It is suggested that raster with hexagonal pattern provide denser scan than grid.}

Calibrations in the previous sequences shall include the following errors:
\begin{itemize}
\item Telescope elevation, and azimuth
\item Temperature
\item Focus
\item ROA
\item Error in the relative position of A\&G cameras to fiber positioner.
\item Error in Cobras' position measured prior to integration at Subaru
\item Error in Catalogue
\item ADC
\item etc…… (TBD)
\end{itemize}

We also check the dependency of the map on wavelength, and check if ADC works correctly, by comparing the observed positions with predicted ones by ADC.

When the second distortion map is made, update ETS using this map.


\begin{itembox}[l]{\suctitle{Success Criteria}}
Second distortion map is obtained with error less than 10um.

\bluetext{Required long time to analyze the data?: Yes. \\
--- It takes time ($\sim$ one month) to improve distortion map, by analyzing raster scan data in various condition.
Also it needed more than one run to check dependency of temperature and so on.
}
\end{itembox}

When this step is completed, the fibers shall point to the target within the accuracy of 10 um.
Then, we verify the instrument performance in the following steps. 