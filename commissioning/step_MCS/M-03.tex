%%%%%%%%%%%%%%%%%%%%%%%%%%%%%%%%%%%%%%%%%%%%
\subsubsection{Check of Basic Functions of MCS on the Telescope [Daytime]}\label{secflow:MCSon}

\milestone{MCS is operated on the telescope for the first time.}

After installation to the telescope, basic functions of MCS are checked again, but on-telescope condition in this sequence.
During installing operation, the power of MCS will be off, so we should turn on the MCS electric devices again.

Then we check the following functions (the same as \ref{secflow:MCSoff}):

\begin{itemize}
\item MCS can read CMOS sensor with expected performance.
	\begin{itemize}
	\item dark level: No detectable dark current at 20--25 [decC] with 1.1 sec. exposure time
	\item noise level: $6.4 e^-$
	\item background level: \redtext{how much??}
	\item shutter: We take several images changing the exposure time to confirm the mechanical shutter works properly.
	\end{itemize}
These values are cited from the CDR slides ({\tt MCS camera system.pdf}).
\item MCS can read environmental and mechanical (?) sensors.
	\begin{itemize}
	\item 7 temperature sensors \redtext{(where)?}
	\item \redtext{flow meter?}
	\end{itemize}
\end{itemize}

After the basic functions are confirmed to work, the calibration data (flat, bias and, dark?:TBC) of CMOS sensor shall be obtained.
The stability of calibration data can be examined using the data in multiple runs.

By sending command from gen2, we will also confirm communication between MCS and Gen2.

\redtext{
According to the MCS delta-CDR, the flat screen is attached MCS flange for obtaining flat-fielding, and 3--4 LEDs are attached near MCS-M1 module.
On the other hand, because the Cassegrain stand-by flange has a space for light source on the top, the calibration data can be obtained during stand-by (TBD).
}

\begin{itembox}[l]{\suctitle{Success Criteria}}
All MCS basic functions are verified --- power, CMOS sensor, telemetry.  \\
The calibration data (bias, flat, and dark?) of CMOS are acquired.

\bluetext{Required long time to analyze the data?: No. \\
--- We can check the functions in real-time}
\end{itembox}

