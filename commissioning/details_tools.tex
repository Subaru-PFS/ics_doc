\subsection{Required Tools}\label{sec:required_tools}
As shown the above, a various tools are requires, some of which is used for scientific operation and other of which is designated for the commissioning (or engineering).
Table \ref{tbl:tools} summarizes input and output of individual tools.

%%%%%%%%%%%%%%%%%%%%%%%%%%%%%%%%%%%%%%%%%%%%
%%% Table of Compliance Matrix


%--------------------------------------------------------------
%  Table: expected runs and nights
%--------------------------------------------------------------
%\setlength{\tabcolsep}{1mm}{
\begin{landscape}
\begin{longtable}{p{42mm}|p{42mm}|p{40mm}|p{50mm}|p{35mm}}
%\begin{center}
\caption{
The list of tools used for PFS commissioning (TBW). 
Column 4 describes a method to see if the success criteria (column 3) is satisfied. 
The tools in the blue row are designated only for the PFS commissioning.}
\label{tbl:tools} 
%\scriptsize
\footnotesize
%\\ \hline
%\multicolumn{4}{c}{ Process Name and Summary} \\ \hline
%Inputs	& Outputs & Success Criteria & Notes \\ \hline \hline
\endhead
%\hline
\endfoot
\hline
%%% M--1, M--3
\multicolumn{5}{c}{ \ref{secflow:MCSoff} and \ref{secflow:MCSon}: Initial check of MCS} \\ \hline
Inputs	& Outputs & Success Criteria & Method & Notes \\ \arrayrulecolor[rgb]{0.63,0.79,0.95}\hline \hline \arrayrulecolor{black}
Gen2 command to acquire MCS image. 	&  Shutter moves (if needed) and MCS image is saved.	& Image is acquired with name and headers along with datamodel, and is archived. & Check if the fine exist in target place. Check the filename and header. & \\  \hline
Gen2 command to open/close MCS shutter. 	&  Shutter moves and its status returns.	& Shutter movement and status are consistent.	& Take image after moving shutter. Alternatively, hear shutter moving using microphone. & 	\\  \hline
Gen2 command to read MCS temperatures. 	&  Values of the temperature sensors return. 	& The temperature values are correct.	& See if the sensor value is reasonable by compare the dome temperature and/or history. & \\  \hline
Gen2 command to read MCS coolant flow. 	&  Value of the flow meter returns.	& The flow value is correct.	& See if the sensor value is reasonable. &  \\  \hline
(Gen2 command to listen microphone?) 	&  Sound from microphone.	& Sound is transferred correctly.	& Make some sound around the microphone and listen to it. & How to transfer audio is under discussion.  \\  \hline
Gen 2 command to execute the above in sequence. 	&  Outputs of individual commands.	& Individual commands succeeded.	&  	\\ \hline \hline
 \rowcolor[rgb]{0.94, 0.97, 1.0}
Gen2 command ``for IIC" to change the telescope Az. after taking image.	&  Image is taken, telescope moves, and the statuses returns. 	&  Az. moves after an MCS exposure.	& See if the telescope moves. &  Demonstration of ``callback" function. \\  \hline
%%% M--4
\multicolumn{5}{c}{ \ref{secflow:prestudy} : Study of the prime focal plane } \\ \hline
Inputs	& Outputs & Success Criteria & Method & Notes \\ \arrayrulecolor[rgb]{0.63,0.79,0.95}\hline \hline \arrayrulecolor{black}
Gen2 command to acquire a series of MCS images ($\sim$ 10 or 20 frames). 	&  MCS images are saved.	& Images are acquired and archived. & See if the images exist in target place. &  \\  \hline
(Gen2) command to measure the spot centroids. 	&  Spots are identified and their centroids are measured.	& Centroids are measured and saved with IDs. & Check the (ICS/MCS) database. &  \\  \hline
Gen2 command to change telescope El. and InR.		&  Telescope moves and its status returns.	& N/A	& N/A & The telescope movement itself is out of the scope of the PFS commissioning.  \\  \hline
 \rowcolor[rgb]{0.94, 0.97, 1.0}
A series of ($\sim$ 100 frames) measured centroids. 	&  Average of the centroids.	& Averages are obtained. & Compare individual data and average for one or two spots. & \\  \hline
Position on MCS.  	&  Position on F3C.	& Coordinates are transformed within accuracy of TBC um (on prime focal plane). & Compare calculated positions to hole positions as designed. & Coordinates transformation function will be developed in analyzing the data. \\  \hline
%...	& ...	& ...	& ...	& ...	& ... \\ \hline \hline
%\end{center}
\end{longtable}
\end{landscape}

