%%%%%%%%%%%%%%%%%%%%%%%%%%%%%%%%%%%%%%%%%%%%
\section{Effect of the Moonlight on the PFS Commissioning}
TBW.

Effect on \\
1.  limiting magnitude of guide star. \\
2. number of bright star for astrometry and raster scan. \\
3. WFC alignment.

%\section{Hexapod Operation}\label{sec:Hexapod}

%The goal of Hexapod operation is to keep the x/y alignment, focus (z), and tilt of the WFC+HSC assembly with respect to the primary mirror.
%The Hexapod operation can depend on EL following a look-up table (see page 19). The z adjustment can also account for the effect depending on telescope truss temperature (Ttruss) via a simple model (details of the model TBC).
%The look-up table comes from a result of Mirror Analysis (MA) with HSC in POpt2 (see page 47), but the Hexapod operation can be optimized for PFS (see page 46).
%The instructions in the look-up table and those from the model with Ttruss can be imperfect, so probably an offset will need to be given based on a focusing operation from time to time during a night.

%\section{Optimizing Hexapod Operation}

%A significant part of the Hexapod operation is determined by the investigations such as MA with HSC, so the result may not be optimal for PFS.
%However PFS can collect information to optimize it using the AG camera images:
%Image quality, focal plane tilt z \& tilt
%Asymmetry in a defocused star image or coma features in a focused star image  x/y (feasibility is under investigation, see a separate report)
%Initially PFS will have a look-up table copied from HSC, but this will be a different entity, so according to the information from the investigation using PFS, the PFS's look-up table can be edited to modify the Hexapod operation. It is also possible to apply offsets from a different command path.


%\section{Mirror Analysis}\label{sec:MA}
%PFS doesn't have capability of executing Mirror Analysis (MA).
%According to the document describing the mirror analysis and pointing analysis for PFS ({\tt TM-N57208}), PFS uses the MA data of HSC.
%PFS uses the same coefficients of primary mirror as those for HSC, while adds differential parameter for coefficients of secondary mirror. 
%When HSC updates the MA, operators can decide whether they also update the coefficients for PFS.

%The best x/y/z position of WFC+HSC assembly with respect to the primary mirror is investigated by observing a bright star using a Shack-Hartmann system onboard HSC. 
%The MA is performed at several EL to have the x/y/z positions as a function of EL. The operation of actuators for primary mirror support is also optimized as a function of EL via MA.
%The MA is performed only at the field center, so cannot investigate focal plane tilt. Instead HSC itself is used to check the focal plane by looking at stellar images across the field of view. This determines the optimal tilt operation of Hexapod.
%The optimization of x/y/z and tilt can be an iterative process: Once the tilt operation is determined, the MA is performed again to see if the best x/y/z position needs to be tweaked, and one may need to come back to tilt optimization.

\section{Summary of Fundamental Functions of Subsystems to be Achieved before PFS Commissioning.}
Table \ref{tbl:subfuncs} lists subsystems functions which should be validated before PFS commissioning.
In column 1 is listed the ID of function.
In column 2 is written brief description of the function.
In column 3 are listed related items in compliance matrix of PFS commissioning (table \ref{tbl:funcs}).
In column 4, checked-mark will be put when the function will be validated.
In column 5 is listed reference in PFS requirements.
In column 6 are written other notes.
%---------------------------------------------------
% Table: Subsystem functions at delivery.
%---------------------------------------------------
%%%%%%%%%%%%%%%%%%%%%%%%%%%%%%%%%%%%%%%%%%%%
%%% Table of Compliance Matrix


%--------------------------------------------------------------
%  Table: expected runs and nights
%--------------------------------------------------------------
%\setlength{\tabcolsep}{1mm}{
\begin{landscape}
\begin{longtable}{r|p{100mm}|p{25mm}|c|p{30mm}|p{50mm}}
%\begin{center}
\caption{
The list of validated functions before the commissioning.}
\label{tbl:subfuncs} 
%\scriptsize
\footnotesize
\\ \hline
No	& Functions & No. of Table \ref{tbl:funcs} & Succ.?  & Req.	& Notes \\ \hline \hline
\endhead
%\hline
\endfoot
%%% PFI
\multicolumn{6}{l}{\hspace{5mm} {\bf PFI (including Cable C)}} \\ \hline
%%% Initial check
P01	& Instrument rotator moves from $-278 \degree$ to $+278 \degree$.	& 8	&	&	& \\ \hline
P02	& Positioning error of A\&C camera is $\sim$ 2.8 um.	& 25	&	&	& CDR \\ \hline
P03 & PFI has 97 fiducial fibers.	&	& 	& {\tt REQ-FOC-L3-267}	& CDR \\ \hline 
P04 & All fiducial fibers can be back-illuminated.	&  28, 30	& 	& {\tt REQ-L3-PFI-008}	& \\ \hline
P05 & Science fibers home position is measured in the accuracy of 50 um PTV.	& 34, 35	& 	&	& Relative position to a fiducial fiber near the center. \\ \hline
P06 & Science fibers rotation center is measured in the accuracy of 50 um PTV.	& 34, 35	& 	&	& Relative position to a fiducial fiber near the center. \\ \hline
P07 & Perimeter fiducial fibers position is measured in the accuracy of 50 um PTV.	& 34, 26	& 	&	& Relative position to a fiducial fiber near the center. \\ \hline
P08 & Field element dots position is measured in the accuracy of 50 um PTV.	& 34, 35	& 	&	& Relative position to a fiducial fiber near the center. \\ \hline
P09 & The position of AG camera fiducial fibers is measured in the accuracy of 10 um PTV.	& 26	& 	&	& Relative to nearest perimeter fiducial fiber. \\ \hline
P10 & The position of AG camera sensors is measured in the accuracy of 10 um PTV.	& 	& 	&	& Relative to nearest perimeter fiducial fiber. \\ \hline
P11 & The focus position of science fibers is measured in the accuracy of 12 um PTV.	& 	& 	&	& Relative to designed focus position. \\ \hline
P12 & The focus position of perimeter fiducial fibers is measured in the accuracy of 12 um PTV.	& 	& 	&	& Relative to designed focus position. \\ \hline
P13 & The focus position of AG camera sensors is measured in the accuracy of 12 um PTV.	& 	& 	&	& Relative to designed focus position. \\ \hline
P14 & The tilt of science fibers is measured in the accuracy of $0.17 \degree$ PTV.	& 	& 	&	& Relative to a fiducial fiber near the center. \\ \hline
P15 & The tilt of perimeter fiducial fibers is measured in the accuracy of $0.17 \degree$ PTV.	& 	& 	&	& Relative to a fiducial fiber near the center. \\ \hline
P16 & The tilt of AG camera sensors is measured in the accuracy of $0.17 \degree$ PTV.	& 	& 	&	& Relative to a COB rear surface. \\ \hline
P17 & P05--P16 are achieved with rotation angle of $0, \pm 60 \degree$ and El. $90,60,30 \degree $	& 	& 	&	& CDR \\ \hline
P18 & The cobra positioner converge to requested x/y position in the accuracy of 10 um PTV.	& 34, 35	& 	&	&  \\ \hline
P19 & P19 is achieved with rotation angle of $0, \pm 60 \degree$ and El. $90,60,30 \degree $	& 	& 	&	& CDR \\ \hline
P21	& AG camera shall acquire image.	& 	& 	& 	& 	\\ \hline
P21	& Fiducial fiber viewing cameras shall acquire image.	& 6	& 	& 	& 	\\ \hline
P22	& Center camera shall acquire image.	& 7	& 	& 	& 	\\ \hline
P21	& PFI shall be operated remotely via MHS.	& 4, 10	& 	& 	& 	\\ \hline
...	& ...	& ...	& ...	& ...	\\ \hline
%%% MCS
\hline
\multicolumn{6}{l}{\hspace{5mm} {\bf MCS}} \\ \hline
M01	& The FoV is large enough to see all science and fiducial fibers at a time.	& 16, 17	&	& {\tt REQ-MET-2} 	& MCS covers 462mm in diameter at the focal plane (CDR).	\\ \hline
M02	& The fiber centroid cam be measured within the error of 5 um (RMS) for all fibers.	& 15, 16, 17, 26	& 	& {\tt REQ-MET-6}	& $<$ 4.46um RMS (CDR)	\\ \hline
M03	& MCS completes one cycle of image acquisition, data read, and measurement within 5 seconds in average. 	& 18 	& 	& {\tt REQ-MET-7}	& $<$ 3 seconds (CDR) 	\\ \hline
M04	& MCS shall be operated remotely via MHS.	& 1,3	& 	& 	& 	\\ \hline
...	& ...	& ...	& ...	& ...	& ...	\\ \hline
%%% SpS
\hline
\multicolumn{6}{l}{\hspace{5mm} {\bf SpS (including Cable A)}} \\ \hline
S01	& The blue detector shall be cooled down to operating temperature ($173 \pm 0.5$ K)	& 	& 	& {\tt SPS-REQ-307}, {\tt SPS-REQ-202}, {\tt SPS-REQ-95}	& 	\\ \hline
S02	& The red detector shall be cooled down to operating temperature ($173 \pm 0.5$ K)	& 	& 	& {\tt SPS-REQ-307}, {\tt SPS-REQ-203}, {\tt SPS-REQ-95}	& 	\\ \hline
S03	& The nir detector shall be cooled down to operating temperature ($110 \pm 0.5$ K)	& 	& 	& {\tt SPS-REQ-307}, {\tt SPS-REQ-204}, {\tt SPS-REQ-95}	& 	\\ \hline
S04	 & All camera units cover 2394 science fibers in their readable area.	& 	& 	& 	& 	\\ \hline
S05	 & SpS shall back-illuminate science fibers.	& 29, 30 	& 	& 	& 	\\ \hline
...	& ...	& ...	& ...	& ...	& ...	\\ \hline
%%% SpS
\hline
\multicolumn{6}{l}{\hspace{5mm} {\bf Cable B}} \\ \hline
F01		& Cable shall deliver 2394 light source from PFI to SpS.	& 14	& 	& {\tt REQ-FOC-L3-267}	& 	\\ \hline
F02		& Cable connection should be monitored remotely. 	& 14	& 	& {\tt REQ-FOC-L3-043}	& 	\\ \hline
...	& ...	& ...	& ...	& ...	& ...	\\ \hline
%...	& ...	& ...	& ...	& ...	& ... \\ \hline \hline
%\end{center}
\end{longtable}
\end{landscape}
