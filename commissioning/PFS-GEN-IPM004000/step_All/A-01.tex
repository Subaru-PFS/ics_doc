%%%%%%%%%%%%%%%%%%%%%%%%%%%%%%%%%%%%%%%%%%%%
\subsubsection{Characterization of PSF and Spectral Distribution [Daytime/Nighttime]}\label{secflow:PSFchar}

\milestone{The light is delivered from the primary mirror to the detectors for the first time.}

In this step, we measure PSFs and the spectral distribution on the detectors.
As twist of fiber changes, PSF depends on telescope positions as well as Cobra position in its patrol area.
This step is similar to step \ref{secflow:SpSchar}, but the data includes all optics and systems; namely, Primary Mirror --- WFC, --- Field Element --- Cable C --- Cable B --- Cable A --- SpS, and all connectors between them.

PSF/spectral distribution image will be sent to 2D DRP team of Princeton University, who will characterize them in detail.
We will check basic properties of PSFs such as FWHM etc., though.

This sequence consist of three step: (1) principal characterization, (2) measurement of sky spectra, and (3) characterization according to Cobra movement.
The arc/flat lamp on PFI is used to measure PSFs/spectral distribution of all fibers.
We therefore also check the uniformity of the calibration lamps, and validate the calibration sequence.

\paragraph{Principal Characterization}
As soon as PFS can deliver light from the primary mirror to the Spectrograph detector, we will take arc and flat.
As of the writing (June 2016), it can be done during PFI check on Telescope (\ref{secflow:PFIon}), because PFI, Cable B and two SMs will be delivered.
In order to look into the PSF, we shall dithered flat as well.
Dark and bias data is needed for calibration.
Check wavelength resolution, wavelength coverage for all fibers.

Once Cobra calibration is carried out, sparse arc and flat shall be taken in order to look into the PSF wing.
How sparse is TBD, but we can estimate AIT data at LAM.
Current plan\footnote{In his draft, he requested to observe semi-crowded fields before we can target objects for demonstration and PR. May be it can be done during later phase of \ref{secflow:1stDM}, if it is acceptable accurate fiber positioning is crucial.} by Robert Lupton from Princeton University (2D-DRP team) is every other fibers and every 10th fibers with shorter and longer integrations.

\paragraph{Measurement of Sky Spectra}
We will take arc and continuum image with changing telescope elevation and/or RoA.
For example,
\begin{itemize}
\item RoA = $-270, -180, -90, -60, 0, +60, +90, +180, +270$ (9 positions)
\item Elevation = $30, 45, 60, 75$ (4 positions)
\end{itemize}
%If we hide every 3 fibers on the detector with the dots, we need 3 types of fiber configuration.
If we take 3 types fiber configuration (e.g. full, a little sparse, very sparse), there are $3 \times 4 \times 9 = 108$ configurations in total.
%Therefore, there are $4 \times 9 = 108$ configurations in total.
Assuming that it takes 5 minutes to configure, and acquire a couple of images, it will take 9 hours for a series of either arc or continuum images.
As time will be limited, we will firstly characterize PSFs in operation RoAs at every elevation angles, and then study other RoAs.
If we take images at 9 RoA positions at El=60 degree, and 3 RoA operational positions at other elevations (18 telescope positions in total), it will take 4.5 hours.
In the current plan, required time is estimated assuming measurement at 18 positions.
Ideally, as sky condition changes during a night, we shall revisit to a given elevation and RoA every hour or so.

By comparing the properties of PSF such as FWHM, we examine dependency and repeatability of the spots on telescope pointing.
%It is difficult to take arc/continuum image using calibration lamps with telescope pointing to other direction than zenith.
%We will use sky as light source for PSF measurement.
%At this moment, it is not necessary to moving fiber positioner, because we can take spectra of somewhere of the sky.
In the PFS operation RoA is limited from $-60 \deg$ to $+60 \deg$, considering that impact of twisted fibers on FRD.
During the commissioning, however, we shall check the PSFs in wider range of RoA.
If the impact is enough low at wider RoA, we can have more flexibility of operation (e.g. constraint due to dots.).

\paragraph{Characterization according to Cobra Movement}
In this sequence, we also examine PSF variance with respect to the Cobra rotation angles ($\theta$, $\phi$).
Here, we use calibration lamps with the telescope pointing at zenith.
If we change Cobra positions by 60 degrees, namely
\begin{itemize}
\item $\theta$ = $0, +60, +120, +180, +240, +300, +360$ ($0, +360$ mean limit angles)
\item $\phi$ = $0, +60, +120, +180$ ($0, +180$ mean limit angles)
\end{itemize}
We will take $3 \times 7 \times 4 = 84$ sets of data, for 3 fiber configurations.
We will take $3 \times 7 \times 4 = 84$ sets of data, for 3 fiber configurations.
It will take 7 hours to acquire a series of arc data.
We shall take images while Cobras are moving.

\bigskip

During the commissioning phase, SMs 1, 2, and 3 will be delivered ahead of the delivery of PFI.
Therefore, we can repeat the same test for the first three SMs when SM4 will be tested.
Alternatively, if we can characterize PSF well for the first SMs, we can limit the PSF test for the last SM, in order to minimize the telescope time.

In either case, the test using sky shall be limited, for example
\begin{itemize}
\item RoA = $-60, 0, +60$ (3 positions)
\item Elevation = $30, 60$ (2 positions)
\end{itemize}
In this case, it will take 0.5 hours for either arc or continuum data.

%\redtext{Measure wavelength resolution, wavelength coverage. 
%Sequence \ref{secflow:PSFchar} can be skipped at least wavelength resolution and coverage for SM3,4, which will be ready in August 2018.}

\paragraph{Designated Tool fot This Process}
Detailed analysis will be done by 2D-DRP team (Princeton).
For PFS commissioning, a tool for measure FWHM is needed.

\begin{itembox}[l]{\suctitle{Success Criteria}}
The dependency of PSF and spectral distribution on Telescope elevation and RoA is measured.

\bluetext{Required long time to analyze the data?: No. \\
--- Detailed PSF characterization is mainly carried out by 2D-DRP team, and we will proceed to the next sequence in parallel with analysis.
}
\end{itembox}