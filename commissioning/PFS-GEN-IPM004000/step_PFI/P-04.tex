%%%%%%%%%%%%%%%%%%%%%%%%%%%%%%%%%%%%%%%%%%%%
\subsubsection{Validation of MCS performance [Daytime]}\label{secflow:MCSperf}

\milestone{MCS takes fiducial/science fiber image for the first time.}

After PFI and MCS are installed to the Prime Focus and the Cassegrain  Focus, respectively, the image quality of the MCS is validated finally, because at this point MCS can take the image of ``fibers" in the entire field of view for the first time.

We check the intensity of the spots of the fiducial fibers and the science fibers, and confirm the S/N meets the requirement ($>$ several hundreds, according to CDR).
We assume the science fibers, even though partly, can be back-illuminated, that is, the Back Illumination Assembly in the SpS subsystem does work (see \ref{secflow:PFIon}, where we can do this step alternatively) \footnote{According to the latest schedule (as of December 2019), Cable B1 and SM1,2, will be delivered before the PFI arrival. We will use B1 and SM1 for this process.}.

At the beginning, we regulate the brightness of back-illumination.
The brightness of back-illuminated fibers are set based on that of the science fibers illuminated in the strobe mode, whose brightness is fixed ($1 \; \mathrm{[W\;m^{-2} \; sr^{-1}]} \pm 20 \%$).
Firstly, we shall regulate the brightness the brightness of the fiducial fibers comparing with that of science fibers illuminated in the strobe mode.
Secondary, we shall adjust the brightness of science fibers back-lit in the normal mode to that of fiducial fibers. 


After we confirm MCS can take fibers image with required S/N, we validate the MCS performance.
%We validate the MCS exposure and centroid measurement sequence; taking image and calculating the positions of fibers with expected time ($<5$ seconds, according to CDR).
MCS is designed to have uniform PSF with FWHM of 8.5um, or 2.6 pixel, in the entire field of view of PFI.
For the accuracy of the position of the spots, the requirement is less than 5um (0.08 pixel) rms, 
The validation will be processed as follows:
\begin{enumerate}
\item Take the image of back-illuminated fiducial and science fibers, and measure the FWHM of the PSFs.
\item Take several fiber images and calculate the positions errors at a given Elevation. 
\item Repeating 1. and 2. at various elevations and rotator angles, test the stability of the image quality and derived centroids.
\end{enumerate}

If we take 50 frames for a given elevation and rotator angle, it will take $\sim$10 minutes (see also \ref{secflow:prestudy}).
Assuming we test at 6 rotator angles (with 60$\degree$ step from EL$=-180\degree$ to 180$\degree$) for every 5 elevation (with 15$\degree$ step from EL$=30\degree$ to 90$\degree$), 5 hours are required for this test.
Hence, 0.5 day is enough for this process, including the time for verifying the S/N of the spots.

\paragraph{Designated Tool for This Process}
Nothing except for statistics of the measured values.
For measurement of centroid and FWHM, we will use MCS software.


\begin{itembox}[l]{\suctitle{Success Criteria}}
MCS can take fiber image and measure fiber positions. \\
The spot size is $\sim$ 8.5um (2.6 pixel) in the entire field of view. \\
Position of all fibers is estimated within the accuracy of 5um.

\bluetext{Required long time to analyze the data?: No. \\
--- We can analyze FWHM and and centroid accuracy on relatively short-timescale.
}
\end{itembox}