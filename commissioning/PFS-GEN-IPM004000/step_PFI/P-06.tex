%%%%%%%%%%%%%%%%%%%%%%%%%%%%%%%%%%%%%%%%%%%%
\subsubsection{Validation of Auto Guide Cameras [Nighttime]}\label{secflow:AGCfunc}
\milestone{AGCs acquire star image for the first time, and the telescope is guided using AGCs for the first time.}

Six Auto Guide cameras on the edge of FoV are used for many purposes:
\begin{enumerate}
\item Acquisition and Guidance of the field 
\item Auto Guiding
\item Focusing
\item Telescope Pointing analysis
\item Alignment of WFC$+$PFI with the prime mirror
\end{enumerate}

To enable these functions, in this process we validate basic functions of AG cameras --- (1) focusing, (2) pixel scale and orientation of each sensor on the sky, and (3) auto guiding, that is, measuring the centroid and sending position error to the telescope.

\paragraph{Focusing}
Firstly, we seek on-focus position.
The focusing procedure is proposed by Jim Gunn (PFI:336)\footnote{As of February 2020, Satoshi Kawanomoto from NAOJ is thinking algorithm too. This process may change following to discussion in the future}.
With a glass plate covering half area of the detector, AG camera has two focus positions.
Hence, the cameras can take out-focus/ in-focus image simultaneously, when the PFI is at focus.
Measuring the spot size of stars at the area with and without glass plate, $s_+$ and $s_-$, we calculate focus positions.
Namely, $\delta f = (s_+^2 - s_-^2)/(4 \alpha \Delta)$, where $\Delta$ is 150 um from the current design.
The parameter $\alpha$ shall be calculated using simulated data (TBC), and in the commissioning sequence, we shall optimize the parameter $\alpha$.
If 4 out of 6 cameras can take pair of stars (in/out), the error of estimation of focus seems $\sim$10um (Jim's document: PFI:336).

To determine $\alpha$, we 
\begin{enumerate}
\item Slew the telescope to a given field.
\item Seek on-focus position changing $z$ direction. 
Here either area with or without position is used.
\item Move focus position by 1.40 mm, then measure spot sizes ($s_+$ and $s_-$).
In the current design, AG camera detectors locate 1.40mm back-ward from the best position of scientific fibers (see reference for {\tt REQ-SYS-1390}).
\item Repeating step 1.--3. at a various telescope position, derive $\alpha$.
\item After deriving $\alpha$, slew telescope to a given field, configure fibers to stars somehow, and seek focus position using AG camera.
Changing the focus position around on-focus position, estimate position accuracy.
By taking spectra, check the consistency of the on-focus position between the science fibers and the AG cameras.
\end{enumerate}

Assuming 10-sec exposure, it will take$\sim$ 10 minutes to execute step 1.--3. for a given field.
Here we set 7 different focus positions.
The same number is used to measuring the focusing accuracy (step 4.).
If we visit three fields every 15$\degree$ from El.=30$\degree$ to E.l=75$\degree$ to derive $\alpha$, it will take 120 minutes.
Another 40 minutes are needed to measure focusing accuracy.
Therefore, $\sim$ 3 hours are required for this step, including time for analysis.
%We slew telescope to a bright stars and let them into the AG camera FoV.

\paragraph{Focal plane characterization}
$z$-position of eash AG camera needs to be checked to make sure every AG camera makes the same plane when we rotate the instrument rotator.
The prcedure is:
\begin{enumerate}
\item Slew the telescope to a given field.
\item Check focus position on each camera.
\item Rotate the instrument rotator and check focus on each camera.
\item Repeat item 3. and compare the planes which each AG camera travels.
\end{enumerate}

Note that this test can be done with data taken below (``Pixel scale and orientation on the sky").

\paragraph{Pixel scale and orientation on the sky}
Secondary, we measure pixel scale and orientation of the AGCs in the following steps.
Because AG cameras locate every 60$\degree$ , we have to measure the orientation individually (see below).

\begin{enumerate}
\item Slew the telescope to the star clusters, and measure the pixel scale by the astrometry of stars on sensors.
\item Put slight offset on the telescope in R.A and Dec. direction individually, and check the orientation of the x/y direction.
(And we can also check pixel scale by comparing two shifted images.)
\end{enumerate}
The derived pixel scale and orientations are implemented to the software module for acquisition and guidance.

Assuming that it takes 5 min for one exposure from slewing telescope to acquiring images, 
it takes 5 x 4 (position) x 6 (sets) = 120 min in total.
Here, 4 positions are needed  to measure the x/y  direction: the origin and with offset of either R.A or Dec..
To measure the direction easily, we will move InR so a given AGC as to have PA of 0$\degree$.
Therefore 6 sets are needed.
Note that if there are any fields for the two diagonal AG cameras to have enough number of stars, we can reduce to 3 sets.
Including the time to execute astrometry and calibration, 150 minutes = 2.5 hours are needed to this test.

%%%%%%%%%%%%%%%%

\paragraph{Auto guiding}
Finally, we validate the AGC functions as the guiding and confirm that the tracking error meets the requirement \redtext{(How much?)}.
For auto guiding, one or two AG camera are used to measure the position error of the telescope, and another camera is used to measure focusing.

Validation is done following standard procedure for Acquisition and Guidance and Auto Guide:
\begin{enumerate}
\item Slew telescope to a given field.
\item Execute Acquisition and Guidance sequence --- measure the position of the star on AG cameras, calculate position error (dRA, dDec) and d(InR), send telescope these errors, and correct the position.
\item Start Auto Guiding --- measure the position of the star on Ag cameras, calculate position error (dRA, dDec) [and d(InR)], send telescope these errors.
At the same time, some AG camera measures focus position.
Focus should be corrected if it gets out-focus, but threshold and frequency is optimized during the commissioning.
\item Keeping guiding for a couple of minutes, measure the guiding error.
\end{enumerate}
%We check that the images is sent to VLAN and Gen2.

%Because commissioning sequences in PFI don't need long-time exposures, the tracking errors on longer timescale ($>$ 10 minutes) shall be checked in  \ref{secflow:PV1}
%Note also that this sequences can be carried out during the next sequences (\ref{secflow:1stDM}), where we take star images with telescope guided.

We check these functions at several telescope positions.
Assuming that it takes 15 min including optimizing the parameters for calculation of the pointing errors, 2 hours are required for this test.
Note that we can also validate auto guiding in later phases and as mentioned above, parameters for correcting focus will be optimized during the commissioning.

\smallskip

Because in this sequence we have star image on AGCs  and control the telescope accordingly, we can validate communication between Gen2/MLP1 and MHS/AGCC.

\paragraph{Designated Tool for This Process}
Spot size and spot position are measured by ICS module (Auto Guide Camera Controller).
Using these values, a script is needed for seeking on-focus position using either in-focus or out-focus area.
Also, a script to measure the orientation of the camera and an analyzer for brightness of spectra are needed.
Astrometry should be done using specified script.

\begin{itembox}[l]{\suctitle{Success Criteria}}
Acquisition and Guidance, focusing, and Auto Guiding can be executed using AGCs.
The focus is measured with the accuracy of $\sim$10 um \\

Pixel scale and orientation of the x/y pixel direction on the sly are measured.
%Error of tracking of telescope is 0.1 arcsec rms(1 minute), 0.2 arcsec rms (10 minutes), and 0.6 arcsec rms (30 minutes).

\bluetext{Required long time to analyze the data?: No. \\
--- We shall analyze the data in real-time.
}
\end{itembox}