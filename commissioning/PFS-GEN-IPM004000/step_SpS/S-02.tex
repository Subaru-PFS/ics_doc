%%%%%%%%%%%%%%%%%%%%%%%%%%%%%%%%%%%%%%%%%%%%
\subsubsection{Characterization of SpS [Daytime]}\label{secflow:SpSchar}

In this process, we characterize the SMs using arc lamps and continuum lamp.
Here we will use Dummy Cable B module [see {\tt DummyCableB-MainDocument.pptx} (PFS-SpS:01680) for details] which is used for the integration by LAM\footnote{To achieve characterization well, lamps with large wavelength coverage would be ideal. We need black body lamp for throughput measurement, but should input the light from slit or GANG connector.}.
Note that LAM also has a plan to characterize each SM during their AIT work\footnote{See their document for validation roadmap ({\tt LAM.PJT.SUM.DS.1024\_1-4\_Verification\_Requirements.pdf}); $\lambda$ coverage --- 2.5.3: checked by centring the image. $R$ --- 2.5.4: extrapolated by FWHM of the spot for testing image quality (maybe only center?).}.} Compared to SM AIT work, we aim to characterize spectra in wider range both spatially and spectrally.
According to the document by Sandrine Pascal (LAM; {\tt PFS-DummyCableB\_V1.0.pdf }), Dummy Cable B has 4 sets of 7 fibers and 8 sets of single fibers.
We can use 5 out of 8 sets at a time.

\paragraph{PSF and spectral distribution measurement}
We take arc-lamp image and continuum lamp image, and measure PSF and spectral distribution of various fibers on the detectors, respectively.
These images are sent to DRP development team, who will analyze them in detail.
\redtext{Fiber configurations for measurement is TBD.}
%\footnote{LAM will use 4 fibers with the option of extra 7 fibers sparsely arranged in the detectors, but is it acceptable for DRP development?
%It seems some room for fiber configurations.
%1--2 fibers per block, for instance,  are too much?
%At least, the PSFs and spectral distribution both centre and edge seems required.
%Besides, how many spots is needed in wavelength direction?
%}.

\paragraph{Spectral resolution}
We take the arc-lamp images, and  measure FWHM of the spots.
Then we estimate spectral resolution including wavelength dependence, and compare with specifications.
The requirement for the spectral resolution ({\tt REQ-SpS-42 (Analysis)}) is:
\begin{itemize}
\item $>$2300 @ 520 nm (blue)
\item $>$2800 @ 810 nm (red, LR),  $>$5000 @ 810 nm (red, MR)
\item $>$4100 @ 1110 um (nir)
\end{itemize}
We will characterize at least 4 fibers at the outer-most and inner-most of up and down areas, and all 3 sets of 7 fibers.

\paragraph{Wavelength coverage}
We take continuum and measure the effective area for signal, and take arc-lamp image and derive pixel-wavelength relation.
Then we estimate wavelength coverage using the relation, and compare with specifications.
The requirement for the wavelength coverage ({\tt REQ-SpS-41 (Analysis)}) is:
\begin{itemize}
\item 380--650 nm (blue)
\item 630--970 nm (red, LR), 710 - 885 nm (red, MR)
\item 940--1260 nm (nir).
\end{itemize}
As spectral resolution, we will characterize at least 4 fibers at the outer-most and inner-most of up and down areas, and all 3 sets of 7 fibers.

\paragraph{Throughput}
Given that throughput will be measured at LAM before shipping to Subaru (discussion in the 8th collaboration meeting in December 2016), this measurement is probably optional.
For the condition of experiment at laboratory is more stable than at the summit.

Nevertheless, in this paragraph the procedure is described for throughput measurement.
We illuminate GANG connector with a black body lamp, and take fiber images.
We count the intensity of the detector, and measure throughput of the SpS itself.
We compare the measured values with specifications, which is estimated by combining the performance of each element (LAM).
If measured throughput is inconsistent with analyzed one, consider possible causes (vignetting, background etc.) 
For this measurement, the stability of the black body lamp (TBD) is needed.

The requirement for the throughput of SpS ({\tt REQ-SpS-48 (Analysis)}) is in Table \ref{tab:SpSthroughput}:
\begin{table}[!ht]
\begin{center}
\caption{}
\label{tab:SpSthroughput}
\begin{tabular}{c|cc}  \hline
 & \multicolumn{2}{c}{Throughput [ \% ]} \\
wavelength &  requirement & goal \\ \hline \hline
380 & 14 & 19 \\
440 & 38 & 49 \\
550 & 39 & 50 \\
650 & 33 & 50 \\
790 & 42 & 56 \\
980 & 36 & 47 \\
1260 & 35 & 45 \\  \hline
\end{tabular}
\end{center}
\end{table}

\paragraph{Designated Tool for This Process}
A tool for FWHS measurement is needed.
For throughput measurement, illuminating system including light source is needed.

\begin{itembox}[l]{\suctitle{Success Criteria}}
SpS is characterized. These characters is consistent with expected by specification.
\bluetext{Required long time to analyze the data?: Yes. \\
--- No, but we don't need to analyze on real time for this process. 
This process doesn't require the telescope, either.
}
\end{itembox}