%%%%%%%%%%%%%%%%%%%%%%%%%%%%%%%%%%%%%%%%%%%%
\subsubsection{First Pass Distortion Map [Night-time]}\label{secflow:1stDM}

In the PFS operation we need to convert sky coordinate (Sky Catalogue) to another convenient to position fibers (F3C).
For this coordinate transformation we have a ``$0^\mathrm{th}$-pass" distortion map from the optical model with the as-built WFC \& model of atmospheric refraction (including the differential effect), prior to the commissioning.

In this process, we calibrate the distortion map purely by the AGCs with no fibers involved (First pass distortion map).
This process comprises two major sequences: calibration of the sky scale, and astrometry on AGCs.

\paragraph{Calibration of the Sky Scale}
Firstly, we calibrate the sky scale on F3C (focal plane).
For this calibration we measure the position of A\&G camera relative to their two fixed fiducial fibers.
%, in order to make the sky-F3C relation consistent between A\&G cameras and the field of view.

We proceed this sequence as follows:
\begin{enumerate}
\item Slew the telescope to have a star in the center of either half area of a given A\&G camera.
To proceed step 2. easily, rotate PFI for the target camera to  locate PA=0.
Note that the center of the A\&G camera cannot be used because of the shade of the glass plate.
%Either side (``inside'' or ``outside") is fine, but we use the same side for all cameras.
%(Here we assume half ``inside" area.)
\item  Move the telescope to have the AG fixed fiducial fiber pointed to the star.
Then execute the raster scan around the fixed fiducial fiber and calibrate the distance between A\&G camera and the fiber.
\item Repeat 1. and 2. for the other half area of the AG camera.
\item Repeat 1.--3. for all fixed fiducial fiber of all 6 A\&G cameras.
\end{enumerate}
Because one A\&G camera has two fiducial fibers on both sides in the azimuth direction, the raster scan should be carried out using both fibers.
Therefore, we shall do $12 \times n$ raster scan in total.

Assuming that we take 9 images for each AG fixed fiducial fiber ($3" \time 3"$ grid), it will take 10 minutes to take a set of raster scan images, including the moving telescope, readout, etc.
Therefore, 10 $\times$ 2 (areas) $\times$ 2 (fibers) $\times$ 6 (cameras) =240 minutes (6 hours) is requires for this sequence.

\redtext{Typical exposure time of fiducial fiber viewing camera.}

\paragraph{Astrometry}
Secondary, we make distortion map by images from the 6 AG cameras.
The total field of view of the A\&G camera is 5.5 arcmin$^2$ $\times$ 6 = 33 arcmin$^2$.
AG camera is lightly out of focus when we focus the telescope; a half region of camera is inside, while the other half outside.
However, the accuracy in calculation centroids of a spot is $\sim$3um, referring estimation by Jim Gunn:{\tt Focus Offsets, Sensitivity, and Star Counts for AG Cameras (ver. 2015.05.18)}, is small enough to make the distortion map (accuracy: $\sim 50$um).
%Note that the registration of images $\sim$ 1.4 deg apart from the field center, should be possible thanks to the off-telescope calibration of A\&G camera positions on F3C.
%This should be useful to give constraints to the models.

We proceed this sequence as follows:
\begin{enumerate}
\item Take deep images ($\sim 60$ sec exposure: TBD) of bright stars with good astrometry.
We take 3--5 images for statistics.
\item Derive distortion of AG cameras by measuring the positions of stars on camera and comparing with catalogue.
Update the distortion map of the entire FoV by interpolating with the new A\&G distortion map.
\item Repeat 1.--2. at various elevation and azimuth and refine the map as a function of El, Az, and ROA
At present, we will visit from EL=30, 45, 60 and 80$\deg$, and 2--4 Az positions at every elevation (TBD).
\end{enumerate}
Using the calibrated data, the transformation from Sky Catalogue ti F3C is refined.
The new distortion map is called ``$1^\mathrm{st}$-pass" distortion map.

Assuming it will take $\sim$ 15 minutes for a given field, 80 fields will be visited in 20 hours.
We will execute this process at two tun.
At the first run, we will calibrate sky scale and do astrometry using $\sim$50 fields to make the first distortion map.
Using the first distortion map, we visit the several fields to check this map and update if needed.
At this point, we can configure the science fibers and take spectra to demonstrate the instrument performance.
So we take fiber spectra for a couple of fields.

%Note: \\
%* AG position at MCS is defined using fiducial fibers for AG cameras \\
%* We need Astrometry catalogue or a single catalogue during the commission (use UCAC4 ?).

\paragraph{Designated Tool for this Process}
A method is needed to analyze AG camera images for astrometry, and analyze raster scan data of the fiver viewing cameras.
\redtext{To do: Check output from fvcactor. If it outputs intensity of the fiber spot, the method will be simple..}

\begin{itembox}[l]{\suctitle{Success Criteria}}
The ``$1^\mathrm{st}$-pass" distortion map is obtained with the accuracy of 50um

\bluetext{Required long time to analyze the data?: Yes. \\
--- It shall takes time (one month?) to astrometry of A\&G Camera,  derive distortion map, measure sky scale on F3C. 
}
\end{itembox}