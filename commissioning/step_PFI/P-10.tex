%%%%%%%%%%%%%%%%%%%%%%%%%%%%%%%%%%%%%%%%%%%%
\subsubsection{First Pass Distortion Map [Night-time]}\label{secflow:1stDM}

In the PFS operation we need to convert sky coordinate (Sky Catalogue) to another convenient to position fibers (F3C).
For this coordinate transformation we have a ``$0^\mathrm{th}$-pass" distortion map from the optical model with the as-built WFC \& model of atmospheric refraction (including the differential effect), prior to the commissioning.

In this step, we calibrate the distortion map purely by the AGCs with no fibers involved (First pass distortion map).
The sequence is as follows:
\begin{enumerate}
%\item Confirm the orientation of the individual images of AG cameras; \\
%The AG camera pixel scale and orientation of x/y-pixel coordinate can be confirmed by taking two images with an offset of the telescope pointing or by centroiding astrometory stars on them.
\item Make distortion map by images from the 6 AG cameras; \\
Take deep images ($\sim 10-30$ sec exposure) of bright stars with good astrometry and derive distortion map.
Derive distortion of AG camera by measuring the positions of stars on camera and comparing with catalogue.
Here, we need special method to analyze the A\&G images.
We update the distortion map of the entire FoV by interpolating with the new A\&G distortion map.
The total field of view of the A\&G camera is (5.5 arcmin$^2$ $\times$ 6)/2 = 16.5 arcmin$^2$.
The division by 2 is because of the step of the A\&G cameras, half region of which has the same focal plane. 
Note that the registration of images $\sim$ 1.4 deg apart from the field center, should be possible thanks to the off-telescope calibration of A\&G camera positions on F3C.
This should be useful to give constraints to the models.
\item We make the map at various elevation and azimuth and refine it as a function of El, Az, and ROA
\item Measure the sky scale on F3C; \\
We calibrate the position of the A\&G cameras relative to their fixed fiducial fibers.
%, in order to make the sky-F3C relation consistent between A\&G cameras and the field of view.
Firstly, we slew the telescope to have a star in the center of an A\&G camera.
Then we move the telescope to have the AG fixed fiducial fiber pointed to the star.
We execute the raster scan around the fixed fiducial fiber and calibrate the distance between A\&G camera and the fiber.
Because one A\&G camera has two fiducial fibers on both sides in the azimuth direction, the raster scan should be carried out using both fibers.
Therefore, we shall do $12 \times n$ raster scan in total.
\end{enumerate}

Using the calibrated data, the transformation from Sky Catalogue ti F3C is refined.
The new distortion map is called ``$1^\mathrm{st}$-pass" distortion map.


%Note: \\
%* AG position at MCS is defined using fiducial fibers for AG cameras \\
%* We need Astrometry catalogue or a single catalogue during the commission (use UCAC4 ?).

\begin{itembox}[l]{\suctitle{Success Criteria}}
The ``$1^\mathrm{st}$-pass" distortion map is obtained with the accuracy of 50um

\bluetext{Required long time to analyze the data?: Yes. \\
--- It shall takes time (one month?) to astrometry of A\&G Camera,  derive distortion map, measure sky scale on F3C. 
}
\end{itembox}