%%%%%%%%%%%%%%%%%%%%%%%%%%%%%%%%%%%%%%%%%%%%
\subsubsection{Telescope Pointing Analysis [Night-time]}\label{secflow:TPA1}

%\milestone{PFS send Az/El offset measured using AGC to the telescope for the first time.}

Although there are originally 1-, 5-, 32-points Telescope Pointing Analysis (TPA) functions on the Subaru telescope, PFS can execute only 1- TPA\footnote{It is noted that 1-TPA for PFS doesn't seem Subaru standard manner.}.
Here, the number shows the numbers of stars to be observed at a given field.
PFS uses the same mount correction coefficients as HSC, and they are updated when HSC executes TPA.
(Note that operator can decide whether to update for PFS.)

By executing 1- TPA, offset angles for Az/EL are estimated.
These offset are used the additional parameters for mount correction coefficients for PFS.

For 1- TPA, the telescope control system drives the sequence as below:
\begin{enumerate}
\item Slew the telescope to set a certain star on the one of the AG camera.
%\item Take its image on the center camera.
%PFI has a small CMOS sensor ($\sim 9" \times 9"$) in the center of the FoV.
%The FoV of the center camera is large enough to acquire a bright star in the FoV even before TPA.
\item Take its image by AG camera.
\item Calculate the Az/El offset, by measuring offset of the star from where it should be.
\item Repeat step 1.--3. for different stars.
We will visit every 30$\degree$ in the azimuth direction and every 15$\degree$ from 30 -- 75$\degree$ in elevation.
In total , Az/El offsets are  measured at 12 $\times$ 4 $=$ 48 field of view.
\item Average offset measured at sequence [3].
Update Az/El offset value of mount correction efficients, as the additional parameter.
\end{enumerate}

Assuming it will take 300 sec to obtain the data in a given field, 4 hours (300 $\times$ 48 seconds) are required.
We derive mount correction efficients in a run, and check them in another run.
In total, it will take 8 hours to complete this process.

\redtext{Check how much does the difference of the scale b/w FoV center and edge effect??}
%The offset derived from this process can be considered from the next telescope pointing. 

%In this step, we start with 1-, 5- TPA.
%The stars are chosen from various areas of sky so that the offset can be characterized as functions of telescope azimuth and elevation.
%Since PFI does not have any camera at the field center\footnote{There are discussion whether we locate a $\sim$ 1mm camera in the center of FoV, instead of fiducial fiber.}, a special care will be necessary to slew the telescope and calculate the offset.
%That is, we should slew telescope w/ offset ($\Delta \alpha$, $\Delta \delta$) in order to have image on the center of AG cameras.
%These offset is derived from $0^\mathrm{th}$-pass distortion map in the first operation (see below), but derived from updated distortion map after commissioning.

%After TPA, we check tracking error of the telescope.

\paragraph{Designated Tool for This Process}
It depends on ICS software (MAC: MLP1 AGCC Converter) has this function (TBC).

\begin{itembox}[l]{\suctitle{Success Criteria}}
Obtain Az/El offset value of mount correction efficients.

\bluetext{Required long time to analyze the data?: No. \\
--- We shall analyze the data in real-time.
}
\end{itembox}