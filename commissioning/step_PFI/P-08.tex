%%%%%%%%%%%%%%%%%%%%%%%%%%%%%%%%%%%%%%%%%%%%
\subsubsection{Refine of the PFI -- MCS relation [Daytime]}\label{secflow:mcs2f3c}
Once the alignment of WFC with the primary mirror is confirmed at a given telescope position (\ref{secflow:WFCTiltShift}), we can confirm and refine PFI--MCS relation using fiducial fibers in this sequence (see also section \ref{sec:coord_calib}).

Prior to the integration, the focal plane from the viewpoint of MCS (F3C--MCSC relation) has been calculated with ZEMAX, using the WFC as-built model.
%Using the ZEMAX calculation, we derive the form of relation in advance
%This relation can be predicted by fitting the relation with a given function.
\redtext{(the function form: TBD)}
Using this formula, we calculate the positions of fixed fiducial fibers on F3C.

The physical positions of fixed fiducial fibers on focal plane, on the other hand, will be measured during the PFI I\&T process at ASIAA before shipment.
By comparing derived positions with the physical positions, we refine the position of the fixed fiducial fibers on F3C.

We will carry out this process as follows:
%At the beginning of this sequence, we use this function.
\begin{enumerate}
\item Take the images of back-lit fixed fiducial fibers with MCS, and transform the centroid on MCSC to F3C.
\item Compare measured fixed fiducial fiber positions with physical positions.
Here, we aim at getting these positions consistent with each other by 10 um RMS (TBC).
If the positions of fixed fiducial fibers are consistent between F3C and physical position, the position on F3C are registered.
The positions with terrible discrepancy shall be excluded from F3C.
\end{enumerate}
%the relation between fiber position on PFI ($x_{PFI}, y_{PFI}$) and on MCS ($x_{MCS}, y_{MCS}$).
%If the transformed positions is different from the original ones on F3C, the function shall be improved.
%We shall derive the PFI--MCS relation at various El. and ROA in order to check the dependency on them, and minimize the error of the function.

%The improved PFI--MCS relations are sent to coordinate transformation system as MCS--F3C transformation function.
At this point, the transforming function from the MCS coordinate to the F3C coordinate is defined.
If we take $\sim$ 10--20 images to measure fixed fiducial fiber position on F3C, $\sim$15minutes  is quite enough. 
Including comparing positions, updating the coordinate transformation (and hence softwares), 1 day is enough for this process.

Once MCS-F3C transformation is confirmed, MPS can send command for moving the Cobras to dot position.
We verify this command, by taking the images with the fibers on-/off- dots after the Cobra calibration \ref{secflow:CobraCal}.

\paragraph{Designated Tool for This Process}
A tool for plotting physical positions and F3C positions, to compare them easily.
Centroids on F3C will be derived in the standard way for coordinates transformation (by Fiber Positioner Sequencer).

\begin{itembox}[l]{\suctitle{Success Criteria}}
PFI--MCS relation is calibrated and MCSC--F3C transformation is confirmed. \\
Fixed fiducial fibers position on F3C is consistent with physical position by 10 um RMS (TBC), 
%MPS command for dot position is validated.

\bluetext{Required long time to analyze the data?: No. \\
--- We shall analyze the data in real-time.
}
\end{itembox}