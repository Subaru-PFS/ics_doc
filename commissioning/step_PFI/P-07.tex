%%%%%%%%%%%%%%%%%%%%%%%%%%%%%%%%%%%%%%%%%%%%
\subsubsection{Check focal plane shift/tilt [Night-time]}\label{secflow:WFCTiltShift}

To have good image quality on the focal plane, WFC (and hence the instrument) need to be aligned in x/y/z and tilt with respect to the telescope primary mirror.
The misalignment of WFC with the primary mirror in x/y causes coma aberration across the field.
That in z defocuses images.
A tilt error induces coma aberration and focal plane tilt.
So it is important to minimize these errors to keep good image quality during observation. 
This alignment is carried out by Hexapod on POpt2 that can adjust the x/y/z and tilt of the assembly of WFC and PFI (see section \ref{sec:Hexapod}).
In this process we optimize the Hexapod position for PFS.

We use out-focused ($\Delta$D5=0.5mm) image of the bright star on AGCs to measure shift and tilt of WFC with respect to M1 and correct them.
The shift and tilt are featured by ellipticity and 3-rd order moment vector, respectively.
Measuring these values, we correct the shift and tilt of WFC.
The goal of correction is that shift is less than 0.5mm, and that tilt is less than 1 arcmin, with which the spot size (RMS) is less than 15 um.
[See Yoko Tanaka's report detailed procedure (WFC\_position.pdf: in Japanese)]

The alignment procedure is as follows:
\begin{enumerate}
\item Check focus, and defocus by $\Delta$D5=0.5mm.
\item Slew the telescope to bright stars.
\item Acquire the image.
To measure tilt and shift of WFC, 6 defocused image are needed on every AG camera.
Since there is a constraint on position of the spot on the camera, and bright stars are needed to achieve S/N$\sim$100, we acquire 6 images by rotating the instrument or moving the telescope.
\item Measure ellipticity and 3-rd order moment vector from 6 images, and estimate correction values.
Apply correction.
\item Repeat 3 and 4 a few times until shift and tilt meets the requirement.
\end{enumerate}

Assuming it takes about 3min including positioning, one iteration needs about 20 min.
According to Yoko Tanaka's study, four iterations on average are needed to alignment.
Therefore, it takes ~80min for one position.

The parameters of Hexapod for alignment is registered as a function or table with respect to the elevation.
Therefore, we measure the Hexapd parameters at several elevations.
Given we measure them 5 positions (e.g., EL=30$\degree$, 40$\degree$, 50$\degree$, 60$\degree$, 75$\degree$) and repeat three times at each elevation, 800 minutes, or 20 hours are needed for the correction.
In the current plan, we will derive the function of Hexapod parameters with 2 sets of measurement at the run 2.
At the next run, we will execute the other 1 set to check the registered function, and modify if needed.
Note that it will be fine visit apart of 5 positions above to check the alignment, but it will be better to modify the functions with measurement at all 5 positions.

\smallskip
\paragraph{Designated Tools for This Process}
This process needs designated scripts
(1) to get 100 x 100 pixels data from FITS image,
(2) to measure shift and tilt and correct them, and
(3) to determine the function of Hexapod's parameters with respect to the elevation, or create tables. 
Yoko Tanaka from Subaru is preparing the script (2), and (3) [TBC].
PFS is supposed to prepare the script (1).

Note that performance of Tanaka algorithm will be tested before PFS engineering runs using HSC, in the S17B semester (run 0).

\begin{itembox}[l]{\suctitle{Success Criteria}}
Misalignment of  WFC and M1 is corrected (shift: 0.5mm and tilt 1 arcmin)

Hexapod position at several elevation are measured and registered as correction parameters.

\bluetext{Required long time to analyze the data?: No. \\
--- We shall analyze the data in real-time to align WFC.
}
\end{itembox}