%%%%%%%%%%%%%%%%%%%%%%%%%%%%%%%%%%%%%%%%%%%%
\subsubsection{Validation of Auto Guide Cameras [Nighttime]}\label{secflow:AGCfunc}
\milestone{AGCs acquire star image for the first time, and the telescope is guided using AGCs for the first time.}


Six Auto Guide cameras on the edge of FoV are used for many purposes:
\begin{enumerate}
\item Acquisition and Guidance of the field 
\item Auto Guiding
\item Focusing
\item Telescope Pointing analysis
\item Alignment of WFC$+$PFI with the prime mirror
\end{enumerate}

To enable these functions, in this phase we validate basic functions of AG cameras --- (1) focusing, (2) pixel scale and orientation of each sensor on the sky, and (3) auto guiding, that is, measuring the centroid and sending position error to the telescope.

\paragraph{Focusing}
Firstly, we seek on-focus position.
We slew telescope to a bright stars and let them into the AG camera FoV.
The we take AG camera image and seek on-focus position.
AG cameras has step of 300um, with which they can take in-focus/ out-focus image simultaneously. 
By measuring the size of these spots, seek focus position.
The focusing procedure is proposed by Jim Gunn (PFI:336), to derive $\alpha$ parameter.
The parameter shall be calculated using simulated data (TBC), and in the commissioning sequence, we shall optimize the parameter $\alpha$.
If 4 out of 6 AGC can take pair of stars (in/out), the error of estimation of focus seems $\sim$10um (Jim's document: PFI:336)

\paragraph{Pixel scale and orientation on the sky}
Secondary, we measure pixel scale and orientation of the AGCs.
We slew the telescope to the star clusters, and measure the pixel scale by the astrometry of stars on sensors.
We then check the orientation of the x/y direction, putting slight offset on the telescope in R.A and Dec. direction individually.
(And we can also check pixel scale by comparing two shifted images.)
The derived pixel scale and orientations are implemented to the software module for acquisition and guidance.

Assuming that it takes 5 min for one exposure from slewing telescope to acquiring images, 
it takes 5 x 3 (position) x 6 (sets) = 90 min in total.
Here, 3 positions are needed  to measure the x/y  direction: one is the origin and the others are with offset of either R.A or Dec..
To measure the direction easily, we will move InR so a given AGC as to have PA of 0 degree.
therefore 6 sets are needed.
If there are any fields for the two opposite AG cameras to have enough number of stars, we can reduce to 3 sets.
Including the time to execute astrometry and calibration, 120min=2hours are needed to this test.

%%%%%%%%%%%%%%%%

\paragraph{Auto guiding}
Finally, we validate the AGC functions as the guiding and confirm that the tracking error meets the requirement \redtext{(How much?)}.
For auto guiding, one or two AG camera are used to measure the position error of the telescope, and another camera is used to measure focusing.

We slew telescope to a given field, and execute Acquisition and Guidance sequence --- measure the position of the star on Ag cameras, calculate position error (dRA, dDec) and d(InR), send telescope these errors, and correct the position.
Then we start auto guiding --- measure the position of the star on Ag cameras, calculate position error (dRA, dDec) [and d(InR)], send telescope these errors.
At the same time, some AG camera measures focus position.
Focus should be corrected if it gets out-focus, but threshold and frequency is optimized during the commissioning.
We measure the guiding error.

%We check that the images is sent to VLAN and Gen2.

%Because commissioning sequences in PFI don't need long-time exposures, the tracking errors on longer timescale ($>$ 10 minutes) shall be checked in  \ref{secflow:PV1}
%Note also that this sequences can be carried out during the next sequences (\ref{secflow:1stDM}), where we take star images with telescope guided.

We check these functions at several telescope positions.
Assuming that it takes 15 min including optimizing the parameters for calculation of the pointing errors, 2 hours are required for this test.
Note that we can also validate auto guiding in later phases and as mentioned above, parameters for correcting focus will be optimized during the commissioning.

\smallskip

Because in this sequence we have star image on AGCs  and control the telescope accordingly, we can validate communication between Gen2/MLP1 and MHS/AGCC.

\begin{itembox}[l]{\suctitle{Success Criteria}}
Acquisition and Guidance, focusing, and Auto Guiding can be executed using AGCs.

Pixel scale and orientation of the x/y pixel direction on the sly are measured.
%Error of tracking of telescope is 0.1 arcsec rms(1 minute), 0.2 arcsec rms (10 minutes), and 0.6 arcsec rms (30 minutes).

\bluetext{Required long time to analyze the data?: No. \\
--- We shall analyze the data in real-time.
}
\end{itembox}