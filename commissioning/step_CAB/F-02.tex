%%%%%%%%%%%%%%%%%%%%%%%%%%%%%%%%%%%%%%%%%%%%
\subsubsection{Confirmation of Fibre-Slit Relationship [Daytime]}\label{secflow:FibID}

In this step, we check fibre-slit relationship.
It is defined in e.g. \\
 http://sumire.pbworks.com/w/file/76743299/FiberMapping.pdf .

Firstly, we move all fibers to be obscured by the dots.
Then move each fiber out of the dot, take back-illuminated fiber image using MCS and identify it.
We identify position on the spectrograph, on the other hand, by taking a spectra of continuum lamp (or dome light).
%We take fiber spectra by SpS and image by MCS under two configurations: with and without hidden by dot.

Note that we will move cobra to dots before calibration, but it should be safe considering that the position will have been calibrated during I\&T at ASIAA\footnote{If the accuracy of calibration will be poor and is risky to move positioners before calibration, we disconnect the Cable B and A, and check identification by illuminating each holes.}.

%We illuminate individual fibers at both female and male gang connectors (connector of cables A and B) and see the position of the fiber on camera and Cobra.
%The position on the camera is measured by taking images by SpS, while the positions on cobra is measured by MCS.

If the relationship is different than the expected one, update the relationship.
It seems simpler to modify the position on the camera, fixing the positioner ID.
%Given that the coordinate system (VMCS) should use cobra IDs and that it would be more complicated to update them, it is rather simple to update IDs on SMs.
%(That is, Cobra IDs should be fixed. )


\begin{itembox}[l]{\suctitle{Success Criteria}}
Fibre position on Cobra and PFI is confirmed
\end{itembox}